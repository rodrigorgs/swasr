% vim: tw=78 encoding=utf8 ts=2 sw=2 expandtab softtabstop=2
\documentclass{article}

\usepackage[utf8]{inputenc}
\usepackage[brazil]{babel}
\usepackage{url}

\title{
{\small Anteprojeto de Tese de Doutorado} \\
Sobre a Adoção de Boas Práticas 
de Desenvolvimento de Software 
em Projetos de Software Livre}
\author{Rodrigo Rocha Gomes e Souza\\
\texttt{rodrigo@dcc.ufba.br}}

\date{Dezembro de 2009}

\begin{document}

\sloppy
\maketitle

% TODO: Agile Practices: http://books.google.com.br/books?hl=pt-BR&lr=&id=9J_0ly01QicC&oi=fnd&pg=PR13&dq=agile+development+assessment&ots=wUzpQHk_03&sig=kM7FGLX3vmn7Z5MxM4RjSn_a99U#v=onepage&q=practices&f=false 

% transferencia de boas práticas dentro de um projeto de software livre. Ver Szulanski
% Ler michlmayr-process_maturity_success 

% Anotações da reunião com Christina
% Semat
% Procurar trabalhos de assessment de boas práticas
% Levantar a bola pra ser um estudo de larga escala

% TODO Enough of processes:
% In short, a practice is a proven way of approaching or addressing a problem.
% It is something that has been done before, can be successfully communicated to
% others, and can be applied repeatedly producing consistent results.

% TODO: Citações a incluir:
% \cite{szulanski1996}
% \cite{jacobson2007}
% \cite{michlmayr2005}

%%%%%%%%%%%%%%%%%%%%%%%%%%%%%%%%%%%%%%%%%%%%%%%%%%%%%%%%%%%%%%%%%%%%%%%%%%%%%
%%%%%%%%%%%%%%%%%%%%%%%%%%%%%%%%%%%%%%%%%%%%%%%%%%%%%%%%%%%%%%%%%%%%%%%%%%%%%

% OBJETIVO: práticas mais realistas OU 
% coordenação do desenvolvimento

\section{Motivação} % QUAL É O PROBLEMA?
% contexto, problema, relevância

% CONTEXTO
No contexto de desenvolvimento de software, boas práticas são estratégias ou
atividades que comprovadamente contribuem para o sucesso de um projeto de
software. A aplicação de boas práticas dentro de uma equipe torna-a mais
competitiva ao equilibrar as diferenças de desempenho entre seus membros. 

A adoção bem sucedida de uma prática deve envolver toda ou boa parte da equipe.
Não obstante, a transferência de práticas entre desenvolvedores pode ser
prejudicada por diversos fatores \cite{szulanski1996}.
%, tais como falta de motivação ou falta de
%experiência 

Projetos de software livre são tipicamente caracterizados pela participação de
voluntários com diferentes níveis de envolvimento, o que é um fator dificulta a
transferência de práticas na equipe. Outro fator é a alta rotatividade de
desenvolvedores \cite{robles2006}. Além disso, comparado a projetos
desenvolvidos por empresas, projetos de software livre geralmente possuem uma
estrutura de controle menos centralizada \cite{raymond2001}, o que reforça a
importância da coordenação e da comunicação entre desenvolvedores na propagação
de boas práticas.

Crowston e Howison \cite{crowston2005} recentemente mostraram que os padrões de
interação entre desenvolvedores variam bastante de projeto para projeto. Alguns
projetos são bastante centralizados em um conjunto pequeno de desenvolvedores
que interagem com boa parte dos demais; outros possuem uma organização mais
descentralizada, nas quais os desenvolvedores se organizam em grupos bem
definidos. Não é claro, no entanto, como práticas de desenvolvimento se
distribuem nessa rede social. 

%A compreensão dos fatores que influenciam a adoção e a transferência de boas
%práticas é, portanto, estudo de uma prática de desenvolvimento. 
Infelizmente, a maior parte do conhecimento disponível sobre os fatores que
influenciam a adoção de práticas é proveniente de observações isoladas e estudos
de caso \cite{crowston2005}. Essas observações, embora de grande valia, são
essencialmente tendenciosas. Uma compreensão mais abrangente dos fatores é
necessária para que se tomem medidas mais efetivas para a adoção de boas
práticas.

%A escolha feita nesta pesquisa de estudar projetos de software livre certamente
%limita a generalidade dos resultados, uma vez que a organização desses projetos
%em geral é diferente da organização de projetos comerciais \cite{raymond2001}.
A opção por estudar projetos de software livre se justifica pela importância que
tais softwares adquiriram em diversos segmentos. Um estudo mostrou que, em 2004,
mais de 1 milhão de desenvolvedores no Estados Unidos estavam trabalhando em
projetos de software livre. Além disso, projetos de software livre dominam o
mercado mundial de servidores web, servidores de e-mail e servidores DNS
\cite{wheeler2007}.

%%%%%%%%%%%%%%%%%%%%%%%%%%%%%%%%%%%%%%%%%%%%%%%%%%%%%%%%%%%%%%%%%%%%%%%%%%%%%%
%%%%%%%%%%%%%%%%%%%%%%%%%%%%%%%%%%%%%%%%%%%%%%%%%%%%%%%%%%%%%%%%%%%%%%%%%%%%%%


\section{Objetivo}

O principal objetivo desta pesquisa é entender como a adoção de boas práticas
em um projeto de software é afetada pela evolução de aspectos ligados ao
processo, ao produto e à organização de desenvolvedores do projeto.
Pretende-se assim identificar em quais contextos determinadas práticas são
efetivamente aplicadas e entender os mecanismos que reforçam a adoção de
práticas de desenvolvimento de software.

Os seguintes aspectos do processo, do produto e da organização social de cada
projeto serão investigados:
\begin{itemize}
  \item idade do projeto;
  \item tamanho do software;
  \item número de desenvolvedores;
  \item número de downloads do software;
  \item atividade em listas de discussão;
  \item concentração do esforço de codificação em um grupo pequeno de
desenvolvedores.
\end{itemize}

% tirei documentação
A análise será focada nas atividades de implementação e controle de qualidade.
Serão investigadas, entre outras, as seguintes práticas:
\begin{itemize}
%  \item uso de testes de unidade automatizados;
  \item implementação de testes de unidade automatizados antes da implementação
  de funcionalidades;
  \item implementação dos testes por pessoas diferentes daquelas que
  implementam funcionalidades;
  \item correção de \emph{bugs} antes da implementação de novas funcionalidades;
  \item integração contínua das mudanças com o repositório de código-fonte;
  \item posse coletiva de código.
  % coding standards
%  \item revisão de código.
\end{itemize}

Espera-se, com este trabalho, chegar a respostas para questões de pesquisa
como as que se seguem:

\begin{itemize}
  \item A adoção de boas práticas é afetada pelo tamanho da equipe? Quais
práticas são mais afetadas?
  \item A adoção de práticas é mais uniforme em projetos mais centralizados?
  \item Existe relação entre a adoção de práticas e o perfil de um
desenvolvedor? Por exemplo, será que desenvolvedores que participam ativamente
de listas de discussão do projeto tendem a seguir mais boas práticas?
  \item A saída de desenvolvedores de um projeto prejudica a adoção de boas
práticas?
\end{itemize}

%Pretende-se também estudar diferenças entre as práticas adotadas por projetos
%com características distintas. Características de projetos a serem estudadas
%incluem grau de centralização, número de desenvolvedores e tamanho do software.
%Esse estudo poderá levar a conclusões do tipo ``a prática \emph{X} não é
%escalável: ela raramente é usada em projetos de grande porte''. 

%Em especial, as seguintes questões merecem FURTHER investigação:
%Ao investigar a resposta dessas questões em diversos projetos ao longo do
%tempo, se pretende chegar a respostas para as seguintes questões de pesquisa:
%\begin{itemize}
%  \item Em que condições a entrada e a saída de desenvolvedores de um projeto afeta o seu processo de desenvolvimento?
%  \item Existem diferenças nos processos aplicados por desenvolvedores com diversos níveis de envolvimento com o projeto?
%  \item Qual é o papel dos desenvolvedores mais envolvidos com um projeto na manutenção de seu processo de desenvolvimento?
%  \item Os subprojetos de um projeto compartilham um mesmo processo?
%***  \item Quais práticas são mais difundidas em projetos de software livre? Quais são menos difundidas?
%  \item Como se diferenciam os processos de projetos bem sucedidos e os processos de projetos que foram descontinuados?
%  \item Dentre os projetos mais bem sucedidos, de que forma os processos se diferenciam?
%  \item Como os processos se alteram nos diversos estágios do desenvolvimento de uma versão do software (alfa, beta, candidata a lançamento, em manutenção)?
%***  \item Existem práticas que são comumente aplicadas em conjunto?
%  \item Os projetos que dizem seguir um modelo de processo estão de fato aplicando as práticas associadas a ele?
%\end{itemize}



%%%%%%%%%%%%%%%%%%%%%%%%%%%%%%%%%%%%%%%%%%%%%%%%%%%%%%%%%%%%%%%%%%%%%%%%%%%%%
%%%%%%%%%%%%%%%%%%%%%%%%%%%%%%%%%%%%%%%%%%%%%%%%%%%%%%%%%%%%%%%%%%%%%%%%%%%%%

\section{Metodologia}

% TODO: projeto piloto, estudo exploratório preliminar

% VIABILIDADE DO ESTUDO
%   Por que software livre? Porque há dados!
%   Restrição a software livre restringe generalidade dos resultados, uma vez que
% em média os processos de sw livre são diferentes de processos de empresas. Mas
% não diminui a importância do estudo, pois sw livre é muito importante!

Muitos projetos de software livre tornam disponíveis publicamente alguns
artefatos produzidos durante o processo de desenvolvimento de software. Esses
artefatos incluem código-fonte e documentação mantidos em sistemas de controle
de versão, bem como relatórios de \emph{bugs} e mensagens de e-mail trocadas
entre desenvolvedores. 

A grande disponibilidade de dados relacionados ao desenvolvimento favorece a
realização de estudos em larga escala. Nesta pesquisa serão utilizados
predominantemente métodos quantitativos de análise de dados para viabilizar o
estudo de uma amostra significativa de projetos de software livre.

A seleção da amostra de projetos será feita com base em informações como número
de desenvolvedores e tamanho do software. Serão extraídos dados de repositórios
de controle de versão, sistemas de acompanhamento de \emph{bugs} e listas de
discussão.
%Ferramentas de análise estática de código serão usadas para extrair
%relações entre os diversos componentes de um sistema de software.

%Os projetos de software e seus desenvolvedores serão caracterizados com base em
%métricas de redes sociais, como centralidade \cite{crowston2005}, e de outras
%métricas aplicadas em estudos sobre software livre \cite{

Características como idade do projeto e número de downloads poderão ser
obtidas de repositórios de projetos de software livre, tais como o
SourceForge.net. Segundo dados
oficiais\footnote{\url{http://sourceforge.net/about}}, em fevereiro deste ano
o repositório contava com cerca de 230 mil projetos registrados.

A identificação das práticas adotadas por um projeto será feita a partir do uso
de técnicas de mineração de processos \cite{rubin2007} sobre o histórico dos
dados extraídos do projeto. Tais técnicas têm como finalidade extrair modelos de
processos a partir de dados temporais.

% TODO: Newman2003: The Structure and Function of Complex Networks
A organização social dos desenvolvedores será estudada através de técnicas de
análise de redes sociais \cite{newman2003}. Essas técnicas permitirão
investigar aspectos como a transferência e a concentração de trabalho entre
desenvolvedores.

Para responder às questões de pesquisa, serão usados métodos de análise
exploratória de dados e inferência estatística, relacionando características de
projetos e de desenvolvedores, o histórico de desenvolvimento e a aplicação de
boas práticas. A partir dos resultados da análise poderão ser realizadas
investigações mais aprofundadas sobre projetos selecionados e comparações com
resultados obtidos em estudos de caso.

%%%%%%%%%%%%%%%%%%%%%%%%%%%%%%%%%%%%%%%%%%%%%%%%%%%%%%%%%%%%%%%%%%%%%%%%%%%%%
%%%%%%%%%%%%%%%%%%%%%%%%%%%%%%%%%%%%%%%%%%%%%%%%%%%%%%%%%%%%%%%%%%%%%%%%%%%%%

\section{Proposta de Cronograma}
\newcommand{\newrow}{\\\hline}
\newcommand{\x}{$\bullet$}

As seguintes atividades serão realizados durante este projeto, de acordo com os
prazos indicados na Tabela \ref{tab:cronograma}:

\begin{enumerate}
  \item \label{prevista:estudos}
    \textbf{revisão bibliográfica};
    %Serão lidos textos sobre engenharia de software
    %experimental, processos e práticas de desenvolvimento de software,
    %desenvolvimento de software livre, mineração de processos e estatística,
    %entre outros. Ao final será produzido um resumo dos tópicos mais relevantes.
%    Produção de artigos. Os artigos serão submetidos para veículos de publicação
%    a serem definidos.
%  \item \label{prevista:qualificacao}
%    \textbf{Realização do exame de qualificação}.
%  \item \label{prevista:proposta}
%    \textbf{Defesa da proposta de tese}.
  \item \label{prevista:preliminares}
    \textbf{planejamento e execução de estudos empíricos preliminares};
  \item \label{prevista:experimentos}
    \textbf{execução de estudos empíricos completos e análise dos resultados};
    %Espera-se produzir pelo menos dois
    %artigos: um com resultados preliminares e outro com os resultados finais dos
    %experimentos.
  \item \label{prevista:artigos}
    \textbf{divulgação científica};
  \item \label{prevista:redacao}
    \textbf{redação e defesa da tese}.
\end{enumerate}

\begin{table}[h]
  \centering
  \begin{tabular}{|c|c|c|c|c|c|c|c|c|} \hline
    Atividade                   & 2010.1 & 2010.2 & 2011.1 & 2011.2  & 2012.1 & 2012.2  & 2013.1 & 2013.2 \newrow
    \ref{prevista:estudos}      & \x     & \x     & \x     &         &        &         &        &        \newrow
    \ref{prevista:preliminares} &        &        & \x     & \x      &        &         &        &        \newrow
    \ref{prevista:experimentos} &        &        &        &         & \x     & \x      &        &        \newrow
    \ref{prevista:artigos}      &        &        &        & \x      &        & \x      &        &        \newrow
    \ref{prevista:redacao}      &        &        &        &         &        & \x      & \x     & \x     \newrow
  \end{tabular}
 \caption{Proposta de Cronograma de Atividades}
 \label{tab:cronograma}
\end{table}

%%%%%%%%%%%%%%%%%%%%%%%%%%%%%%%%%%%%%%%%%%%%%%%%%%%%%%%%%%%%%%%%%%%%%%%%%%%%%
%%%%%%%%%%%%%%%%%%%%%%%%%%%%%%%%%%%%%%%%%%%%%%%%%%%%%%%%%%%%%%%%%%%%%%%%%%%%%

\bibliographystyle{plain}
\bibliography{doutorado}

\end{document}
