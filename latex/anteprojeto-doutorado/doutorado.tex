% vim: tw=78 encoding=utf8 ts=2 sw=2 expandtab softtabstop=2
\documentclass{article}

\usepackage[utf8]{inputenc}
\usepackage[brazil]{babel}
\usepackage{url}

\title{
{\small Anteprojeto de Tese de Doutorado} \\
Sobre a Adoção de Boas Práticas 
de Desenvolvimento de Software 
em Projetos de Software Livre}
\author{Rodrigo Rocha Gomes e Souza\\
\texttt{rodrigo@dcc.ufba.br}}

\date{Dezembro de 2009}

\begin{document}

\sloppy
\maketitle

% TODO: Agile Practices: http://books.google.com.br/books?hl=pt-BR&lr=&id=9J_0ly01QicC&oi=fnd&pg=PR13&dq=agile+development+assessment&ots=wUzpQHk_03&sig=kM7FGLX3vmn7Z5MxM4RjSn_a99U#v=onepage&q=practices&f=false 

% transferencia de boas práticas dentro de um projeto de software livre. Ver Szulanski
% Ler michlmayr-process_maturity_success 

% Anotações da reunião com Christina
% Semat
% Procurar trabalhos de assessment de boas práticas
% Levantar a bola pra ser um estudo de larga escala

% TODO Enough of processes:
% In short, a practice is a proven way of approaching or addressing a problem.
% It is something that has been done before, can be successfully communicated to
% others, and can be applied repeatedly producing consistent results.

% TODO: Citações a incluir:
% \cite{szulanski1996}
% \cite{jacobson2007}
% \cite{michlmayr2005}

%%%%%%%%%%%%%%%%%%%%%%%%%%%%%%%%%%%%%%%%%%%%%%%%%%%%%%%%%%%%%%%%%%%%%%%%%%%%%
%%%%%%%%%%%%%%%%%%%%%%%%%%%%%%%%%%%%%%%%%%%%%%%%%%%%%%%%%%%%%%%%%%%%%%%%%%%%%

% OBJETIVO: práticas mais realistas OU 
% coordenação do desenvolvimento

\section{Motivação} % QUAL É O PROBLEMA?
% contexto, problema, relevância

% CONTEXTO
No contexto de desenvolvimento de software, boas práticas são estratégias ou
atividades que comprovadamente contribuem para o sucesso de um projeto de
software \cite{jacobson2007}. A aplicação de boas práticas dentro de uma equipe
torna-a mais competitiva ao equilibrar as diferenças de desempenho entre seus
membros.  Não obstante, a transferência de práticas entre desenvolvedores pode
ser prejudicada por diversos fatores, como falta de motivação ou de experiência
\cite{szulanski1996}.
%, tais como falta de motivação ou falta de experiência 

Projetos de software livre são tipicamente caracterizados pela participação de
voluntários com diferentes níveis de envolvimento, o que é um fator que
potencialmente dificulta a transferência de práticas na equipe. Outro fator é a
alta rotatividade de desenvolvedores \cite{robles2006}. Além disso, comparado a
projetos desenvolvidos por empresas, projetos de software livre geralmente
possuem uma estrutura de controle menos centralizada \cite{raymond2001}, o que
reforça a importância da coordenação e da comunicação entre desenvolvedores na
propagação de boas práticas.

Projetos de software livre têm adquirido importância em diversos segmentos. Um
estudo mostrou que, em 2004, mais de 1 milhão de desenvolvedores no Estados
Unidos estavam trabalhando em projetos de software livre. Além disso, projetos
de software livre dominam o mercado mundial de servidores web, servidores de
e-mail e servidores DNS \cite{wheeler2007}.

%Crowston e Howison \cite{crowston2005} recentemente mostraram que os padrões de
%interação entre desenvolvedores variam bastante de projeto para projeto. Alguns
%projetos são bastante centralizados em um conjunto pequeno de desenvolvedores
%que interagem com boa parte dos demais; outros possuem uma organização mais
%descentralizada, nas quais os desenvolvedores se organizam em grupos bem
%definidos. 

%Dada a importância de projetos de software livre e o papel da disseminação de
%boas práticas para o sucesso de um projeto, é 

Infelizmente, a maior parte do conhecimento disponível sobre os fatores que
influenciam a adoção de práticas em projetos de software livre é proveniente de
observações isoladas e estudos de caso \cite{crowston2005}. Essas observações,
embora de grande valia, são essencialmente tendenciosas. Uma compreensão mais
abrangente dos fatores é necessária para que se tomem medidas mais efetivas para
a adoção de boas práticas. 

%%%%%%%%%%%%%%%%%%%%%%%%%%%%%%%%%%%%%%%%%%%%%%%%%%%%%%%%%%%%%%%%%%%%%%%%%%%%%%
%%%%%%%%%%%%%%%%%%%%%%%%%%%%%%%%%%%%%%%%%%%%%%%%%%%%%%%%%%%%%%%%%%%%%%%%%%%%%%

\section{Objetivo}

O principal objetivo desta pesquisa é entender como a adoção de boas práticas
em um projeto de software é afetada pela evolução de aspectos ligados ao
processo, ao produto e à organização de desenvolvedores do projeto.
Pretende-se assim identificar em quais contextos determinadas práticas são
efetivamente aplicadas e entender os mecanismos que reforçam a adoção de
práticas de desenvolvimento de software.

Os seguintes aspectos do processo, do produto e da organização social de cada
projeto serão investigados:
\begin{itemize}
  \item idade do projeto;
  \item tamanho do software;
  \item número de desenvolvedores;
  \item número de downloads do software;
  \item atividade em listas de discussão;
  \item concentração do esforço de codificação em um grupo pequeno de
desenvolvedores.
\end{itemize}

% tirei documentação
A análise será focada nas atividades de implementação e controle de qualidade.
Serão investigadas, entre outras, as seguintes práticas:
\begin{itemize}
%  \item uso de testes de unidade automatizados;
  \item implementação de testes de unidade automatizados antes da implementação
  de funcionalidades;
  \item implementação dos testes por pessoas diferentes daquelas que
  implementam funcionalidades;
  \item correção de \emph{bugs} antes da implementação de novas funcionalidades;
  \item integração contínua das mudanças com o repositório de código-fonte;
  \item posse coletiva de código.
  % coding standards
%  \item revisão de código.
\end{itemize}

Espera-se, com este trabalho, chegar a respostas para questões de pesquisa
como as que se seguem:

\begin{itemize}
  \item A adoção de boas práticas é afetada pelo tamanho da equipe? Quais
práticas são mais afetadas?
  \item A adoção de práticas é mais uniforme em projetos mais centralizados?
  \item Existe relação entre a adoção de práticas e o perfil de um
desenvolvedor? Por exemplo, será que desenvolvedores que participam ativamente
de listas de discussão do projeto tendem a seguir mais boas práticas?
  \item A saída de desenvolvedores de um projeto prejudica a adoção de boas
práticas? De que forma?
\end{itemize}

O conhecimento produzido por esta pesquisa poderá ser usado para se adquirir
maior controle sobre as práticas adotadas em um projeto, tornando o seu
desenvolvimento mais previsível.

%%%%%%%%%%%%%%%%%%%%%%%%%%%%%%%%%%%%%%%%%%%%%%%%%%%%%%%%%%%%%%%%%%%%%%%%%%%%%
%%%%%%%%%%%%%%%%%%%%%%%%%%%%%%%%%%%%%%%%%%%%%%%%%%%%%%%%%%%%%%%%%%%%%%%%%%%%%

\section{Metodologia}

% TODO: projeto piloto, estudo exploratório preliminar
% TODO: SourceForge.

% VIABILIDADE DO ESTUDO
%   Por que software livre? Porque há dados!
%   Restrição a software livre restringe generalidade dos resultados, uma vez que
% em média os processos de sw livre são diferentes de processos de empresas. Mas
% não diminui a importância do estudo, pois sw livre é muito importante!

Pretende-se estudar projetos de software livre hospedados no repositório
SourceForge.net, que abriga cerca de 230 mil
projetos\footnote{\url{http://sourceforge.net/about}}. O SourceForge.net
disponibiliza publicamente, para cada projeto, informações sobre o número de
\emph{downloads} e idade do projeto, além de sistema de controle de versão,
sistema de rastreamento de \emph{bugs} e listas de discussão. 

A grande disponibilidade de dados relacionados ao desenvolvimento favorece a
realização de estudos em larga escala. Nesta pesquisa serão utilizados
predominantemente métodos quantitativos de análise de dados para viabilizar o
estudo de uma amostra significativa de projetos de software livre. Espera-se
assim obter conclusões generalizáveis sobre o uso de boas práticas em projetos
de software livre.

A princípio serão realizados estudos de caso com um pequeno conjunto de
sistemas, em caráter exploratório, a fim de se adquirir familiaridade com as
informações disponíveis e com ferramentas de extração e análise de dados. As
questões de pesquisa serão refinadas de acordo com observações realizadas nessa
etapa.

A seguir será realizado um estudo em larga escala, considerando uma amostra de
projetos estratificada de acordo com dados como número de desenvolvedores e
tamanho do software. Serão extraídos dados de repositórios de controle de
versão, sistemas de acompanhamento de \emph{bugs} e listas de discussão.

A identificação das práticas adotadas por um projeto será feita a partir do uso
de técnicas de mineração de processos \cite{rubin2007} sobre o histórico dos
dados extraídos do projeto. Tais técnicas têm como finalidade extrair modelos de
processos a partir de dados temporais.

A organização social dos desenvolvedores será estudada através de técnicas de
análise de redes sociais \cite{newman2003}. Essas técnicas permitirão
investigar aspectos como a transferência e a concentração de trabalho entre
desenvolvedores.

Para responder às questões de pesquisa, serão usados métodos de análise
exploratória de dados e inferência estatística, relacionando características de
projetos e de desenvolvedores, o histórico de desenvolvimento e a aplicação de
boas práticas. A partir dos resultados da análise poderão ser realizadas
investigações mais aprofundadas sobre projetos selecionados e comparações com
resultados obtidos em estudos de caso.

%%%%%%%%%%%%%%%%%%%%%%%%%%%%%%%%%%%%%%%%%%%%%%%%%%%%%%%%%%%%%%%%%%%%%%%%%%%%%
%%%%%%%%%%%%%%%%%%%%%%%%%%%%%%%%%%%%%%%%%%%%%%%%%%%%%%%%%%%%%%%%%%%%%%%%%%%%%

\section{Proposta de Cronograma}
\newcommand{\newrow}{\\\hline}
\newcommand{\x}{$\bullet$}

As seguintes atividades serão realizados durante este projeto, de acordo com os
prazos indicados na Tabela \ref{tab:cronograma}:

\begin{enumerate}
  \item \label{prevista:estudos}
    \textbf{revisão bibliográfica};
  \item \label{prevista:preliminares}
    \textbf{realização de estudos de caso};
  \item \label{prevista:experimentos}
    \textbf{realização de um estudos em larga escala};
  \item \label{prevista:artigos}
    \textbf{divulgação científica};
  \item \label{prevista:redacao}
    \textbf{redação e defesa da tese}.
\end{enumerate}

\begin{table}[h]
  \centering
  \begin{tabular}{|c|c|c|c|c|c|c|c|c|} \hline
    Atividade                   & 2010.1 & 2010.2 & 2011.1 & 2011.2  & 2012.1 & 2012.2  & 2013.1 & 2013.2 \newrow
    \ref{prevista:estudos}      & \x     & \x     & \x     &         &        &         &        &        \newrow
    \ref{prevista:preliminares} &        &        & \x     & \x      &        &         &        &        \newrow
    \ref{prevista:experimentos} &        &        &        &         & \x     & \x      &        &        \newrow
    \ref{prevista:artigos}      &        &        &        & \x      &        & \x      &        &        \newrow
    \ref{prevista:redacao}      &        &        &        &         &        & \x      & \x     & \x     \newrow
  \end{tabular}
 \caption{Proposta de Cronograma de Atividades}
 \label{tab:cronograma}
\end{table}

%%%%%%%%%%%%%%%%%%%%%%%%%%%%%%%%%%%%%%%%%%%%%%%%%%%%%%%%%%%%%%%%%%%%%%%%%%%%%
%%%%%%%%%%%%%%%%%%%%%%%%%%%%%%%%%%%%%%%%%%%%%%%%%%%%%%%%%%%%%%%%%%%%%%%%%%%%%

\bibliographystyle{plain}
\bibliography{doutorado}

\end{document}
