\documentclass{article}

\newcommand{\newrow}{\\\hline}
\newcommand{\x}{$\bullet$}

\usepackage[utf8]{inputenc}
\usepackage[brazil]{babel}

\newcommand{\review}[2]{\textcolor{red}{#1}\textcolor{blue}{#2}}

\title{
{\small Anteprojeto de Tese de Doutorado}
\\
Processos de Desenvolvimento em \\
Software: Discurso e Prática
%
%Processos de Desenvolvimento em \\
%Projetos de Software Livre
}
\author{Rodrigo Rocha Gomes e Souza\\
\texttt{rodrigo@dcc.ufba.br}
}

\date{Dezembro de 2009}

\begin{document}

\sloppy
\maketitle
%\newpage
%\tableofcontents
%\newpage

%%%%%%%%%%%%%%%%%%%%%%%%%%%%%%%%%%%%%%%%%%%%%%%%%%%%%%%%%%%%%%%%%%%%%%%%%%%%%
%%%%%%%%%%%%%%%%%%%%%%%%%%%%%%%%%%%%%%%%%%%%%%%%%%%%%%%%%%%%%%%%%%%%%%%%%%%%%

% TODO: onion-like structure, hierarchical structure, core developers 

%   Processo afeta qualidade do software
%   Employee turnover is known to be high in the traditional software industry
% since many years ago. (rotatividade)
%   Turnover afeta processo?


\section{Objetivo}

O principal objetivo deste trabalho é traçar a evolução dos processos de
desenvolvimento e da estrutura social de projetos de software livre. Serão
investigadas as causas de mudanças nos processos ao longo do ciclo de vida do
software. Além disso, serão evidenciadas diferenças e semelhanças entre os
processos praticados em projetos de diversos portes, propósitos e níveis de
maturidade.

A análise será focada nas atividades de implementação, documentação e controle
de qualidade. Serão respondidas, para cada projeto, questões como as que se
seguem:

\begin{itemize}
  \item São usados testes automáticos?
  \item Os testes são planejados e escritos antes da implementação das funcionalidades correspondentes?
  \item Os defeitos são corrigidos antes da implementação de novas funcionalidades?
  \item A pessoa que escreve os testes para uma funcionalidade é a mesma que implementa a funcionalidade?
  \item Modificações no código-fonte são integradas frequentemente ao repositório do projeto?
  \item Há um registro sistemático da descoberta e da resolução de defeitos no software?
  \item São seguidas convenções de codificação?
  \item Em que momento a documentação do sistema é escrita?
  \item Os papéis de programador, testador e documentador são exercidos por pessoas diferentes ou os papéis se misturam?
  \item Existe posse de código ou o código-fonte é revisado por diversos programadores?
  \item De que forma se dá a transferência de trabalho entre desenvolvedores?
\end{itemize}

Ao investigar a resposta dessas questões em diversos projetos ao longo do
tempo, se pretende chegar a respostas para as seguintes questões de pesquisa:

\begin{itemize}
  \item Em que condições a entrada e a saída de desenvolvedores de um projeto afeta o seu processo de desenvolvimento?
  \item Existem diferenças nos processos aplicados por desenvovedores com diversos níveis de envolvimento com o projeto?
  \item Qual é o papel dos desenvolvedores mais envolvidos com um projeto na manutenção de seu processo de desenvolvimento?
  \item Os subprojetos de um projeto compartilham um mesmo processo?
  \item Quais práticas são mais difundidas em projetos de software livre? Quais são menos difundidas?
  \item Como se diferenciam os processos de projetos bem sucedidos e os processos de projetos que foram descontinuados?
  \item Dentre os projetos mais bem sucedidos, de que forma os processos se diferenciam?
  \item Como os processos se alteram nos diversos estágios do desenvolvimento de uma versão do software (alfa, beta, candidata a lançamento, em manutenção)?
  \item Existem práticas que são comumente aplicadas em conjunto?
  \item Os projetos que dizem seguir um modelo de processo estão de fato aplicando as práticas associadas a ele?
\end{itemize}

%%%%%%%%%%%%%%%%%%%%%%%%%%%%%%%%%%%%%%%%%%%%%%%%%%%%%%%%%%%%%%%%%%%%%%%%%%%%%
%%%%%%%%%%%%%%%%%%%%%%%%%%%%%%%%%%%%%%%%%%%%%%%%%%%%%%%%%%%%%%%%%%%%%%%%%%%%%

\section{Motivação}

Há muito se discutem formas de tornar o processo de desenvolvimento de software
mais previsível, aumentando a produtividade da equipe e a qualidade do
software. Dessa discussão surgiram modelos de processos e cartilhas de boas
práticas de desenvolvimento.

Existe, no entanto, uma lacuna entre a teoria e a prática do
desenvolvimento de software \cite{glass1996}. A aplicação de processos e práticas é
dificultada por uma série de questões na vida real, desde a escassez de
programas que os suportem até os altos custos de implantação.

Projetos de software livre tornam disponíveis publicamente alguns artefatos
produzidos durante o processo de desenvolvimento, como código-fonte e relatórios
de \emph{bugs}, o que os torna bons candidatos para o estudo empírico da
aplicação de processos e práticas de desenvolvimento.

Pesquisa sobre software livre tem se concentrado em estudos de caso
\cite{mockus2002,capiluppi2007}, com algumas exceções. Crowston e Howison
\cite{crowston2005} estudaram padrões de comunicação em sistemas de relatório de
bugs de 120 projetos de software livre e encontraram grande variedade entre os
projetos. Krishnamurty \cite{krishnamurthy2002} estudou 100 projetos populares e
percebeu que boa parte só possuía um desenvolvedor.

%Embora pesquisas recentes revelem os processos praticados por alguns projetos
%de software livre em determinado período , não se tem uma descrição do
%cenário mais abrangente, que considera uma amostra significativa de projetos ao
%longo do tempo. 

% TODO: desenvolver a ideia
Tal descrição permitiria a identificação de dificuldades e limitações na
aplicação de processos e práticas, contribuindo para a proposta de processos e
práticas mais realistas.
% Lições práticas a serem estudadas na teoria

Muitos projetos de software livre tornam disponíveis publicamente alguns
artefatos produzidos durante o processo de desenvolvimento de software. Esses
artefatos incluem código-fonte e documentação mantidos em sistemas de controle
de versão, bem como relatórios de bugs e mensagens de e-mail trocadas entre
desenvolvedores. Técnicas de mineração de processos e estatística descritiva
viabilizam a realização desta pesquisa com grandes amostras de projetos de
software.

%%%%%%%%%%%%%%%%%%%%%%%%%%%%%%%%%%%%%%%%%%%%%%%%%%%%%%%%%%%%%%%%%%%%%%%%%%%%%
%%%%%%%%%%%%%%%%%%%%%%%%%%%%%%%%%%%%%%%%%%%%%%%%%%%%%%%%%%%%%%%%%%%%%%%%%%%%%

\section{Metodologia}

Esta pesquisa consistirá de um estudo abrangente, envolvendo um grande número
de projetos de software livre, e de um estudo mais aprofundado sobre um número
reduzido de projetos. Os dois estudos cumprem papéis complementares no
entendimento da aplicação de processos em projetos reais.

Para o estudo abrangente será selecionada uma amostra significativa de projetos
de software livre hospedados em um repositório de código-fonte a ser escolhido,
como o SourceForge. A amostra será estratificada de acordo com aspectos como
idade do projeto, número de desenvolvedores e número de linhas de código-fonte.

Serão analisados relatórios de bugs registrados em um sistema de gerenciamento
de bugs, mensagens de e-mail arquivadas em listas de discussão e arquivos
mantidos sob controle de versão, como documentação e código-fonte. Ferramentas
para a extração desses dados já foram usadas em pesquisas anteriores [], com
bons resultados.

A análise dos dados no estudo mais abrangente será feita de forma automática
através de técnicas de mineração de processos. Essa abordagem foi escolhida por
tornar viável a análise de grandes volumes de dados. A mineração de processos
tem como finalidade de extrair modelos de processos a partir de registros de
eventos. Eventos que poderão ser extraídos dos dados coletados incluem a
modificação de um arquivo fonte, o registro da resolução de um bug e o envio de
uma mensagem à lista de discussão do projeto.

%O algoritmo alfa da mineração de processos é capaz de extrair de uma lista de
%eventos um modelo de processo representado como uma rede de Petri. Perguntas
%sobre o modelo extraído podem ser respondidas com o auxílio de checadores de
%fórmulas da lógica temporal linear (LTL).

Inicialmente será feita uma análise exploratória sobre uma amostra reduzida de
projetos de software com os métodos e ferramentas escolhidos. Essa análise terá
por objetivo investigar como os dados devem ser pré-processados e identificar
limitações da abordagem.

A amostra reduzida de projetos de software também será alvo de um estudo mais
detalhado. Informações adicionais sobre o processo desses projetos serão
coletadas através de questionários e entrevistas com desenvolvedores. Quando
for possível, as informações fornecidas pelos desenvolvedores serão
confrontadas com as informações extraídas automaticamente. 


%%%%%%%%%%%%%%%%%%%%%%%%%%%%%%%%%%%%%%%%%%%%%%%%%%%%%%%%%%%%%%%%%%%%%%%%%%%%%
%%%%%%%%%%%%%%%%%%%%%%%%%%%%%%%%%%%%%%%%%%%%%%%%%%%%%%%%%%%%%%%%%%%%%%%%%%%%%

\section{Proposta de Cronograma}

\begin{table}[h]
  \centering
  \begin{tabular}{|c|c|c|c|c|c|c|c|c|} \hline
    Atividade                   & 2010.1 & 2010.2 & 2011.1 & 2011.2  & 2012.1 & 2012.2  & 2013.1 & 2013.2 \newrow
    \ref{prevista:projeto}      &        &        &        &         &        &         &        &        \newrow
    \ref{prevista:estudos}      &        &        &        &         &        &         &        &        \newrow
    \ref{prevista:artigos}      &        &        &        &         &        &         &        &        \newrow
    \ref{prevista:qualificacao} &        &        &        &         &        &         &        &        \newrow
    \ref{prevista:proposta}     &        &        &        &         &        &         &        &        \newrow
    \ref{prevista:redacao}      &        &        &        &         &        & \x      & \x     & \x     \newrow
    \ref{prevista:defesa}       &        &        &        &         &        &         &        & \x     \newrow
  \end{tabular}
 \caption{Proposta de Cronograma de Atividades}
 \label{tab:cronograma}
\end{table}

%%%%%%%%%%%%%%%%%%%%%%%%%%%%%%%%%%%%%%%%%%%%%%%%%%%%%%%%%%%%%%%%%%%%%%%%%%%%%
%%%%%%%%%%%%%%%%%%%%%%%%%%%%%%%%%%%%%%%%%%%%%%%%%%%%%%%%%%%%%%%%%%%%%%%%%%%%%

\bibliographystyle{plain}
\bibliography{doutorado}

\end{document}
