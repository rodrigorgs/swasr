% (5-20)

\chapter{Estudo sobre Algoritmos de Agrupamento} \label{cap:estudo}

\begin{section}[Introdução]

No capítulo anterior foi mostrado que pelo menos três modelos geram redes organizadas em módulos que se assemelham a redes de software. Tal resultado favorece o uso dos modelos para avaliar algoritmos de agrupamento de software. Neste capítulo, um dos modelos é usado em um estudo experimental realizado com o propósito de entender melhor três algoritmos de agrupamento estudados no contexto de recuperação de arquitetura.

O estudo experimental foi dividido em duas partes, com dois objetivos distintos:
\begin{enumerate}
	\item comparar o desempenho dos algoritmos de agrupamento sob o critério da semelhança dos agrupamentos encontrados pelos algoritmos com agrupamentos de referência;
	\item entender como o desempenho dos algoritmos é afetado por parâmetros que descrevem cada rede.
\end{enumerate}

Para este estudo foram escolhidos os algoritmos apresentados no Capítulo \ref{cap:agrupamento}: ACDC, Bunch e algoritmos hierárquicos aglomerativos (ligação simples, SL, e ligação completa, CL). Para cada um dos algoritmos aglomerativos, foram estudadas duas alturas de corte, $0,75$ e $0,90$. No total são 4 configurações de algoritmos aglomerativos, que serão referenciadas como SL75, SL90, CL75 e CL90. Para os algoritmos Bunch e ACDC foram escolhidas as configurações padrão das implementações dos autores dos algoritmos. As 6 configurações escolhidas para este estudo são idênticas àquelas estudadas por Wu, Hassan e Holt \cite{Wu2005}.

A fim de simplificar as análises, o estudo experimental foi realizado com redes geradas por apenas um dos modelos apresentados anteriormente, o BCR+. A escolha se deve à familiaridade do autor com as propriedades do modelo. Foram usadas as 9500 redes geradas no experimento de avaliação de software-realismo, descritas na Seção \ref{sec:parametros}.
	
\end{section}

\begin{section}[Comparação de Algoritmos]

O primeiro experimento teve como objetivo comparar o desempenho dos algoritmos de agrupamento, medido como o valor de MoJoSim (ver Seção \ref{sec:fundamentos-avaliacao}) entre os agrupamentos encontrad
	
\end{section}


---
Objetivos
Planejamento (método)
Resultados






Caracterização do experimento segundo framework de Basili.

Método:
Design fatorial

..
..
..

Resultados:

..
 
--------------

Teste de Mann-Whitney (independente)
Teste de Wilcoxon [signed rank] (pareado)

Algoritmos:
SL75, CL75, SL90, CL90, ACDC, Bunch
Justificativa: Os mesmos usados por Wu e por Andritsos.

Modelo: BCR+
Justificativa: Só havia tempo para um modelo.
Optamos pelo BCR+ porque, por construção, é mais similar a sistemas de software, sob o ponto de vista de que o grafo de dependências entre módulos não é um grafo completo.
LFR foi descartado por ter combinações infactíveis.
CGW foi descartado por ter muitos parâmetros, o que aumenta o custo do experimento.


Wu:
Algoritmos (do melhor para o pior em authoritativeness):
CL90, CL75, Bunch, ACDC, SL75, SL90

Andritsos: (resultados inconclusivos. Ignorar.)