% Avaliação da ("softwareness") Redes de Software e outros tipos de redes
% complexas (5-10)
%	
%		 * as redes geradas são verossímeis?

\chapter{Avaliação dos Modelos}
% Adaptar do artigo do CSMR 2010

Três modelos apresentados no capítulo anterior --- BCR+, LFR e CGW --- geram redes que podem ser usadas para testar algoritmos de agrupamento de software. Para aumentar a confiança de que os resultados dos testes com redes sintéticas podem ser extrapolados para redes de software, no entanto, é preciso avaliar se, com uma escolha adequada de parâmetros, os modelos são capazes de gerar redes sintéticas que se assemelham a redes de software. 

Para 

Já se sabe que os três modelos geram redes que, a exemplo de redes de software, são livres de escala. Isoladamente, no entanto, esta propriedade é insuficiente para provar a hipótese, uma vez que muitas redes livres de escala conhecidas não são redes do software (por exemplo, redes biológicas). Assim, 



Este capítulo descreve um experimento realizado no contexto desta pesquisa, com o propósito de identificar quais modelos apresentados no capítulo anterior são capazes de gerar redes que se assemelham a redes de software.

Se as redes geradas não se assemelharem a redes de software, no entanto, a validade 


de redes estruturadas em módulos que foram apresentados no capítulo anterior --- BCR+, LFR e CGW --- geram redes

A hipótese de pesquisa deste trabalho é a de que pelo menos um dos modelos apresentados é capaz de produzir redes que se assemelham a redes de software.
...

A maior parte dos modelos apresentados no capítulo anterior geram redes que, a exemplo de redes de software, são livres de escala. Isoladamente, no entanto, esta propriedade é insuficiente para provar a hipótese, uma vez que muitas redes livres de escala conhecidas não são redes do software (por exemplo, redes biológicas). A métrica de similaridade deve, portanto, ser capaz de distinguir entre redes software e redes de outros domínios.

Seguindo o trabalho de Milo et al. \cite{Milo2004}, neste trabalho 


 
% Descrição do experimento segundo o arcabouço de Basili.
% 
% Descrição geral do experimento: similaridade entre redes sintéticas e redes de software usando tríades.
% 
% PARTE 1
% 
% Similaridade entre duas redes.
% 
% Conjuntos de dados.
% 
% Métrica de softwareness e avaliação através de precisão e cobertura.
% 
% PARTE 2
% 
% Avaliação dos modelos
% 
% Escolha dos parâmetros
% 
% Resultados