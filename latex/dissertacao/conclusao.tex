% 2-10 páginas

%   (Limitações)
%   (Lições aprendidas)
% Recapitulação
% Resumo das contribuições
% Trabalhos futuros

\chapter{Conclusão} \label{cap:conclusao}

Neste trabalho foi apresentada uma nova abordagem para avaliação de algoritmos de agrupamento de software, baseada na simulação de modelos de redes organizadas em módulos. Mostrou-se, através de um experimento, que pelo menos três modelos de redes são capazes de gerar redes que se assemelham a redes de dependências estáticas entre classes em programas orientados a objetos. A partir deste resultado, foi feito um estudo sobre algoritmos de agrupamento de software usando a abordagem apresentada no trabalho.

Este trabalho não diminui a importância de se avaliar algoritmos de agrupamento usando redes extraídas de sistemas de software reais analisadas por especialistas humanos. Ainda assim, o uso de redes de software sintéticas de forma complementar pode ser vantajoso no estudo dos algoritmos. Primeiramente, o uso de modelos permite a criação de conjuntos de teste extensos. Além disso, as redes são criadas de maneira controlada, de acordo com parâmetros pré-definidos, o que torna possível estudar o comportamento dos algoritmos com diversos valores de parâmetros.

O uso de conjuntos de dados sintéticos, produzidos por modelos, é comum na pesquisa de sistemas distribuídos e redes, mas ainda pouco explorado na pesquisa de engenharia de software. O impacto deste trabalho pode ir além da comunidade de agrupamento de software, uma vez que muitas tarefas de engenharia reversa usam dependências entre entidades de software para extrair informações sobre sistemas.

As contribuições desta pesquisa para o estado da arte podem ser assim resumidas:

\begin{itemize}
	\item identificação de dois modelos de redes organizadas em módulos, CGW e LFR, adequados para o teste de algoritmos de agrupamento;
	\item adaptação de um modelo de redes, o BCR, para produzir redes organizadas em módulos;
	\item fornecimento de evidências empíricas de que certos modelos geram redes software-realistas, isto é, redes que se assemelham a redes extraídas de sistemas de software;
	\item identificação de valores para os parâmetros dos três modelos que induzem a geração de redes software-realistas;
	\item ... [TODO: complementar com as contribuições da avaliação de algoritmos de agrupamento]
\end{itemize}

As principais contribuições desta pesquisa foram documentadas no artigo ``Modular Network Models for Class Dependencies in Software'', aceito para publicação no 14º Congresso Europeu sobre Manutenção de Software e Reengenharia (CSMR 2010). Há planos de escrever um outro artigo com os resultados completos da pesquisa.

%%% LIMITAÇÕES

Naturalmente, este trabalho possui algumas limitações que poderão ser endereçadas em trabalhos futuros. A representação usada para sistemas de software --- grafo orientado --- é bastante simples e abstrai aspectos de sistemas de software que podem ser importantes, como a diferenciação de tipos de entidades (classes, atributos, métodos, arquivos fonte etc.). Além disso, o grafo orientado não diferencia diversos tipos de relacionamento entre entidades, como chamada de método, leitura de atributo etc. Em vez disso, todos os relacionamentos são abstraídos sob a forma do relacionamento de dependência.

O foco deste trabalho foi o agrupamento de entidades de software em módulos de forma a minimizar dependências entre módulos. Existem, no entanto, outros critérios para agrupar entidades de software. Por exemplo, algumas pesquisas se preocupam em estudar o agrupamento de entidades que implementam funcionalidades relacionadas. Os modelos estudados neste trabalho não incorporam o conceito de funcionalidade e, portanto, não são adequados para se estudar agrupamentos de entidades de software sob o ponto de vista funcional.

% Limitações
% 
% Trabalhos futuros
%   Refinar modelos, com informações hierárquicas, tipos de vértices, múltiplas arestas com tipos e peso...
%   Refinar modelos, adicionando a dimensão da evolução
%   Pesquisar outros problemas, além de clustering de software, que possam se beneficiar de modelos baseados nos modelos aqui apresentados.