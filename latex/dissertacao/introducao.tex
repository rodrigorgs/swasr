% (5-10) páginas
%
%   Objetivos geral e específicos
%   Resultados esperados
%   Limitações do trabalho
%   Métodos
%   Justificativa
%   Descrição dos demais capítulos

\chapter{Introdução}

\begin{section}{Motivação}
	% MOTIVAÇÃO / JUSTIFICATIVA

	% Estou puxando a brasa para engenharia reversa. Será que é necessário? Não é possível generalizar isso para engenharia de software?

	As dependências entre entidades presentes no código-fonte de sistemas de software (classes, métodos, procedimentos, variáveis etc.) fornecem pistas sobre propriedades de sistemas de software, tais como facilidade de compreensão, facilidade de manutenção, facilidade de teste e facilidade de reuso. 
	% Não surpreende que
	A análise de tais dependências é essencial para o cálculo de diversas métricas de código-fonte, bem como para a realização de diversas atividades de engenharia reversa.

	Engenharia reversa inclui ``qualquer método voltado para recuperar conhecimento sobre um sistema de software existente para apoiar a execução de uma tarefa de engenharia de software'' \cite{Tonella2007}. Nos métodos de engenharia reversa, as dependências podem ser usadas para auxiliar, por exemplo, a recuperação da arquitetura, a detecção de clones ou a localização de conceitos dentro de um sistema de software. 

	Como em muitas outras disciplinas, a maturidade da disciplina de engenharia reversa requer que os métodos propostos sejam avaliados empiricamente. Um estudo realizado sobre artigos sobre engenharia reversa publicados de 2002 a 2005 revelou que 25\% dos artigos não apresentam qualquer forma de avaliação empírica e, dentre os demais artigos, 70\% apresentam apenas estudos de caso e relatos de experiência.

	A escassez de experimentos controlados em engenharia reversa pode ser explicada, em parte, pelo alto custo envolvido na realização de experimentos de qualidade. Em muitos casos, os experimentos precisam envolver grupos de desenvolvedores balanceados segundo o grau de experiência e replicação. O custo de experimentação elevado, no entanto, não é exclusividade da engenharia reversa: ele ocorre também em áreas da computação como redes e sistemas distribuídos.

	% Clarificar: modelos estatísticos? modelos computacionais? modelos estocásticos? modelos para simulação? modelagem científica?
	% NÃO: modelos formais, modelos descritivos, modelos determinísticos
	Uma abordagem empregada quando os experimentos controlados são caros é a simulação de modelos. Na área de redes e sistemas distribuídos, é frequente o uso de ambientes de simulação de redes, que podem incorporar desde modelos de falhas de hardware até modelos de comportamento de usuários. % \cite{XXX}.
	% An Integrated Experimental Environment for Distributed Systems and Networks. 
	% comportamento: http://www.foibg.com/ijita/vol15/ijita15-1-p11.pdf
	%modelos de interação, modelos de distribuição de carga, 

	Na engenharia de software, no entanto, a abordagem de simulação é pouco usada. As aplicações se concentram na simulação de processos de software. Surpreendentemente, não há modelos para dependências entre entidades de sistemas de software.	
\end{section}

\begin{section}{Objetivos}
	O objetivo desta pesquisa é descobrir e avaliar modelos de dependências entre entidades do código-fonte de sistemas de software com a finalidade de apoiar simulações em engenharia de software (em particular, engenharia reversa).
	
	Para atingir o objetivo de pesquisa é necessário cumprir os seguintes objetivos específicos:
	
	\begin{enumerate}
		\item descobrir modelos que podem ser interpretados como modelos de dependências entre entidades de código-fonte;
		\item avaliar os modelos de acordo com a similaridade entre a estrutura das dependências produzidas pelos modelos e a estrutura das dependências que são extraídas de sistemas de software reais;
		\item realizar uma prova de conceito para demonstrar a viabilidade da abordagem de simulação para a avaliação de técnicas e ferramentas que se baseiam na análise de dependências.
	\end{enumerate}
	
	A fim de melhor representar a forma como sistemas de software são idealizados e construídos, optamos por considerar modelos em que as entidades de código-fonte estão organizadas em módulos. % ficou muito seco? desculpa esfarrapada?
	
\end{section}

\begin{section}{Métodos e Resultados}
	
	A teoria das redes complexas estuda métodos para analisar redes (ou grafos) encontradas nos mais diversos domínios, como redes sociais, redes de computadores e redes metabólicas. A aplicação de tais métodos levou à descoberta de propriedades comuns a redes estudadas em diversos domínios, bem como modelos de redes que incorporam essas propriedades.
	
	Dado que as dependências em um sistema de software formam uma rede, é possível aplicar nesta pesquisa alguns dos métodos, modelos e descobertas da teoria das redes complexas, os quais são apresentados no Capítulo \ref{cap:redes}.
	
	Foram encontrados na literatura dois modelos de redes organizadas em módulos, denominados CGW e LFR, os quais podem ser interpretados como modelos de dependências entre entidades de software. Esses modelos são apresentados no Capítulo \ref{cap:redes}. Um terceiro modelo, o BCR+, foi desenvolvido no contexto desta pesquisa. O modelo BCR+ é descrito em detalhes no Capítulo \ref{cap:bcr}.
	
	A avaliação dos modelos, descrita no Capítulo \ref{cap:avaliacao}, teve como base a simulação dos três modelos. As redes produzidas pelos modelos foram comparadas com redes de dependências extraídas de sistemas de software escritos em uma linguagem de programação orientada a objetos. Os resultados indicam que, com uma escolha adequada de parâmetros, os três modelos produzem redes que se assemelham a redes de dependências entre entidades de software (ao menos no caso de linguagens de programação orientadas a objetos).
	
	Para a prova de conceito foi escolhido o problema de recuperação de arquitetura de software. Algoritmos de agrupamento usados na atividade de recuperação de arquitetura foram aplicados a redes de dependências geradas pelo modelo BCR+ e os resultados foram analisados. No Capítulo \ref{cap:estudo} é feita uma introdução ao problema de recuperação de arquitetura e, a seguir, são apresentados os métodos e resultados da prova de conceito.
	
	No Capítulo \ref{cap:conclusao} são apresentadas as conclusões, contribuições e limitações deste estudo.
	
	% Como resultado, foram encontrados dois modelos de redes apropriados para representar dependências entre entidades, e um modelo que foi criado a partir da adaptação de um modelo existente. O novo modelo é apresentado em detalhes no Capítulo ....
	% 
	% A avaliação dos modelos foi realizada através da simulação dos modelos e posterior uso de métodos asd.sd..asd.... Linguagens OO apenas. A avaliação e os resultados são apresentados no Capítulo ....
	% 
	% Para a prova de conceito foi escolhido o problema de recuperação de arquitetura. A prova de conceito .. no Capítulo ....
	
\end{section}
