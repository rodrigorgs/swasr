\chapter{Avaliação de Modelos de Redes} \label{cap:avaliacao}

\begin{section}{Introdução}

Nesta seção é descrito um experimento que mostra que os três modelos de redes organizadas em módulos --- BCR+, CGW e LFR --- geram redes software-realistas. O experimento consiste em gerar redes usando diversas combinações de parâmetros dos três modelos, e então classificar cada rede como software-realista ou não software-realista, usando o modelo de classificação de redes definido no capítulo anterior. 

	Etapas: seleção de parâmetros ...

O número de vértices foi fixado em 1.000 e os valores dos demais parâmetros foram variados em passos discretos. 

%No total, foram geradas 9.500 redes com o modelo BCR+, 38.790 redes com o modelo CGW e 1.296 redes com o modelo LFR.

\begin{subsection}{Seleção de Parâmetros} \label{sec:parametros}

Para o modelo BCR+, foram escolhidos grafos de módulos extraídos a partir de dependências entre arquivos JAR de 5 sistemas de software: GEF (2 módulos), iBATIS (4 módulos), MegaMek (8 módulos), findbugs (16 módulos) e zk (32 módulos). Como muitos dos arquivos JAR foram concebidos para serem reusados em projetos distintos, eles são uma boa aproximação do conceito de módulo. Para os demais parâmetros, os seguintes valores foram escolhidos:

\begin{itemize}
	\item $p_1, p_2, p_3 \in \{0,0; 0,2; 0,4; 0,6; 0,8; 1,0\}$, com $p_1 + p_2 + p_3 = 1$ e $p_1 + p_2 > 0$ (do contrário a rede jamais alcançaria 1.000 vértices);
	\item $\delta_{in}, \delta_{out} \in \{0, 1, 2, 3, 4\}$;
	\item $\mu \in \{0,0; 0,2; 0,4; 0,6\}$ (valores altos foram evitados a fim de ignorar redes com módulos fortemente acoplados).
\end{itemize}

Combinando-se todas as possíveis atribuições de valores a parâmetros dentro dos domínios escolhidos chega-se a 9.500 configurações possíveis.
%Com essa escolha, há um total de 9.500 combinações de valores de parâmetros.
%Para cobrir todas as combinações de valores de parâmetros, foram geradas 9.500 redes com o modelo BCR+.

Para o modelo CGW, os seguintes valores de parâmetros foram escolhidos:

\begin{itemize}
	\item $p_1, p_2, p_3, p_4 \in \{0,0; 0,2; 0,4; 0,6; 0,8; 1,0\}$, com $p_1 + p_2 + p_3 + p_4 = 1$ e $p_1 > 0$ (do contrário a rede jamais alcançaria 1.000 vértices);
	\item $e_1, e_2, e_3, e_4 \in \{1, 2, 4, 8\}$ (com a restrição de que $e_i$ não varia quando $p_i = 0$, o que não faria sentido);
	\item $\alpha \in \{-1, 0, 1, 10, 100, 1000\}$
	\item $m \in \{2, 4, 8, 16, 32\}$.
\end{itemize}

O total de combinações, neste caso, é 38.790. O número elevado se deve à grande quantidade de parâmetros a serem combinados.

Além disso, considerou-se que a rede inicial é formada por dois vértices contidos em um mesmo módulo, juntamente com uma aresta bidirecional que liga os vértices.

No caso do modelo LFR, os seguintes valores foram escolhidos para os parâmetros:

\begin{itemize}
	\item parâmetro de mistura: $\mu \in \{0,0; 0,2; 0,4; 0,6\}$;
	\item expoente da distribuição de graus: $\gamma \in \{2,18; 2,70; 3,35\}$;
	\item expoente da distribuição de tamanhos de módulos: $\beta \in \{0,76; 0,99; 1,58\}$;
	\item grau médio: $k \in \{5, 10, 15, 25\}$;
	\item grau máximo: $max_k \in \{58, 157, 482\}$;
	\item tamanho do menor módulo: $min_m \in \{1, 10, 273\}$.
\end{itemize}

O total de combinações, neste caso, é 1.296.

Para chegar aos valores para os parâmetros do modelo LFR, foram analisadas métricas de redes de software com cerca de 500 a 2.000 classes. Para cada métrica foram identificados os valores mínimo, mediano e máximo; esses foram os valores usados nos parâmetros correspondentes.

%Para cobrir todas as combinações de valores de parâmetros, foram geradas 1.296 redes com o modelo LFR.

\end{subsection}

\begin{section}{Geração de Redes}
	Foi gerada uma rede para cada combinação de valores para os parâmetros de cada modelo. % TODO: continuar...
\end{section}

\begin{subsection}{Resultados}

Usando o modelo de classificação descrito no capítulo anterior, cada rede sintética foi classificada como software-realista ou não software-realista. Os resultados estão condensados na Tabela \ref{tab:results}.

% TODO: refazer esta tabela
\begin{table}
\caption{Resultados da classificação de redes sintéticas}
\centering
\begin{tabular}{|l|l|}
\hline
Modelo & Redes classificadas \\ & como software-realistas \\
\hline 
\hline
BCR+ & 21.18\% \\ % 2012 / 9500
\hline
CGW  & 19.40\% \\  % 7524 / 38790
\hline
LFR  & 31.25\% \\ %  405 / 1296
\hline
\end{tabular}
\label{tab:results}
\end{table}

Todos os modelos geraram redes software-realistas e redes não software-realistas. A proporção de redes software-realistas foi maior que 19\% em todos os casos, descartando a possibilidade de que esse resultado tenha sido obtido por acaso. (A proporção exata para cada modelo não deve ser não deve ser interpretada como medida de qualidade, pois com esses resultados não é possível determinar se um modelo é melhor do que os outros.)
% Para afirmar isso seria necessário repetir o experimento com redes de Erdos-Renyi

Naturalmente, esse resultado tem pouco valor prático se não for estabelecida uma relação entre os valores dos parâmetros usados na geração de uma rede e a classificação da rede. Na prática é importante saber quais valores de parâmetros tendem a gerar redes software-realistas.

Para ajudar a descobrir essa relação, foi utilizado o algoritmo 1R \cite{OneR} da mineração de dados. O algoritmo analisa, para cada rede, os parâmetros usados na sua geração e a sua classificação, e então encontra uma regra que relaciona o valor de um único parâmetro com a classificação da rede. Regras encontradas pelo 1R podem ser avaliadas de acordo com a sua acurácia, isto é, a proporção de redes corretamente classificadas.

As regras encontradas pelo algoritmo 1R são exibidas na Tabela \ref{tab:rules}. As regras são bastante simples e, portanto, fáceis de seguir. (Essa característica do 1R foi o que motivou a sua escolha em detrimento de outros algoritmos de mineração de dados.) Apesar da simplicidade, as regras encontradas possuem uma acurácia de cerca de 80\% para todos os modelos.

% regra obtida a partir dos conjuntos completos
% acurácia, precisão e cobertura obtidos a partir de 10-fold stratified cross-validation
% --
% BCR+
%  regra: p1 >= 0.5
%  acurácia: 83.7904%
%  TP: 5947, FP: 1006, FN: 525, TN: 1967
%  precisão: 85.5314252840501%
%  cobertura: 91.8881334981459%
% --
% CGW
%  regra: e3 >= 3
%  acurácia: 76.623%
%  TP: 11278, FP: 3780, FN: 5287, TN: 18441
%  precisão: 74.8970646832249%
%  cobertura: 68.0833081798974%
% --
% LFR
%  regra: expdegree >= 3.025
%  acurácia: 78.3763
%  TP: 579, FP: 277, FN: 0, TN: 425
%  precisão: 67.6401869158878%
%  cobertura: 100%

\begin{table}
\caption{Regras para prever a classificação de uma rede sintética.}
\centering
\begin{tabular}{|l|l|l|}
% \hline
% Modelo & Regra & Acurácia & Precisão & Cobertura \\
% \hline 
% \hline
% \multirow{2}{*}{BCR+}
%      & $p_1 \ge 0.5 \Rightarrow \mbox{rede software-realista}$ & \multirow{2}{*}{82.4\%} & \multirow{2}{*}{82.4\%} & \multirow{2}{*}{82.4\%} \\ 
%      & $p_1 < 0.5 \Rightarrow \mbox{rede não software-realista}$ & \\ 
% \hline
% \multirow{2}{*}{CGW}
%      & $p_1 \ge 0.5 \Rightarrow \mbox{rede software-realista}$ & \multirow{2}{*}{82.3\%} \\  
%      & $p_1 < 0.5 \Rightarrow \mbox{rede não software-realista}$ & \\  
% \hline
% \multirow{2}{*}{LFR}   
%      & $\gamma < 2.44 \Rightarrow \mbox{rede software-realista}$ & \multirow{2}{*}{78.9\%} \\ 
%      & $\gamma \ge 2.44 \Rightarrow \mbox{rede não software-realista}$ & \\ 
% \hline
\end{tabular}
\label{tab:rules}
\end{table}

% TODO: Mastigar. Exemplo: BCR+: p1 >= 0.7, p2 <= 0.3, p3 <= 0.3, mu <= 0.6, din <= 8, ...

\end{subsection}

\end{section}

\begin{section}{Conclusão}
	
	Os modelos CGW, LFR e BCR+ podem gerar, a depender dos valores atribuídos a seus parâmetros, redes software-realistas, isto é, redes que se assemelham a redes de dependências estáticas extraídas de programas escritos em Java. Esse resultado foi obtido em um experimento no qual as redes geradas pelos modelos foram comparadas a redes extraídas de 65 sistemas de software escritos em Java. Ademais, foram identificadas regras práticas capazes de prever, com 80\% de acurácia, se uma atribuição de valores a parâmetros de um modelo resulta em uma rede software-realista.
	
	 A comparação entre redes foi realizada através da métrica de similaridade que se baseia no perfil de concentração de tríades (PCT) das redes. A métrica de similaridade foi, antes, validada com um conjunto de 66 redes que não pertencem ao domínio de software. A métrica se mostrou adequada para diferenciar essas redes de redes de software com mais de 95\% de cobertura e precisão.
	
\end{section}
 
