\documentclass{article}
\usepackage[utf8]{inputenc}
\usepackage{graphicx}
\usepackage{natbib}

\begin{document}
\title{Artigo 1}
\author{Rodrigo R. G. e Souza}
\maketitle

% Ordem de escrita
%   Resultados
%   Discussão
%   Introdução
%   Materiais e Metodos
%   Abstract
%   Título

\begin{abstract}

goals, results, and the main conclusions of your study

Kent Beck's sentences: The first states the problem. The second states why the problem is a problem. The third is my startling sentence. The fourth states the implication of my startling sentence.

\end{abstract}

\section{Introdução} % why

% why you have investigated the question
% 
% how it relates to earlier research that has been done in the field
% 
% 1. Open with two or three sentences placing your study subject in context
% 2. Follow with a description of the problem and its history, including previous research
% 3. Describe how your work addresses a gap in existing knowledge or ability (here's where you'll state why you've undertaken this study). 
% 4. State what information your article will address. 

É difícil obter muitos dados, grandes sistemas. Difícil achar sistema com arquitetura conhecida.

embora existam alguns trabalhos avaliando algoritmos de recuperação de arquitetura, os resultados experimentais não dão pistas sobre por que os algoritmos são bons em uns critérios e ruins em outros, quais são as coisas que influenciam o desempenho do algoritmo.


\section{Métodos e Materiais} % how

particular techniques used and why, if relevant

modifications of any techniques; be sure to describe the modification

assumptions underlying the study 

statistical methods, including software programs 

\section{Resultados} % what was found

  * present results clearly and logically
  * avoid excess verbiage
  * consider providing a one-sentence summary at the beginning of each paragraph if you think it will help your reader understand your data 

\section{Discussão} % why it's significant

how useful this technique is: how well did it work, what are the benefits and drawbacks, etc

This section centers on speculation

Focus your discussion around a particular question or hypothesis

Trabalhos futuros: explorar métricas de arquitetura, avaliar algoritmos de clustering, considerar outros modelos.

\bibliographystyle{apalike}
\bibliography{complex-networks,rodrigo-mestrado}

\end{document}
