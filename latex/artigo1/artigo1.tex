\documentclass{acm_proc_article-sp}
\usepackage[utf8]{inputenc}
\usepackage[T1]{fontenc}
\usepackage[brazil]{babel}
\usepackage{graphicx}

% Ordem de escrita
%   Resultados
%   Discussão
%   Introdução
%   Materiais e Metodos
%   Abstract
%   Título


% == PLANEJAMENTO DO EXPERIMENTO ==
% 
% INGREDIENTES
%
%   Dois sistemas, um pequeno (?) e um maior (IRPF)
%   Modelos de geração de redes com estrutura de módulos embutida: Rodrigo2008 e Lancichinetti
%   Modelos sem estrutura de módulos: barabasi, bollobas, configuration model.
%
% MODO DE PREPARO
%
%   Analisar métricas dos sistemas (distribuição de graus, distribuição do coeficiente de clustering, correlação de graus).
%   Extrair a arquitetura dos sistemas: quais são os módulos e quais módulos se relacionam. Analisar a distribuição dos tamanhos dos módulos.
%   "Tunar" os parâmetros dos modelos para gerar redes com as mesmas métricas dos sistemas reais
%   Rodar os modelos diversas vezes com os parâmetros encontrados para gerar diversas redes para cada sistema.
%   Comparar redes sintéticas com as redes reais correspondentes através da distância entre redes de Garcia.
% 
% RESULTADOS
%
%   Modelos com estrutura de módulos embutida resultam em redes mais parecidas com as redes reais do que os modelos sem módulos? (Em outras palavras: a informação sobre tamanhos dos módulos e a maneira como os módulos se ligam realmente são uma vantagem?)
%   As redes sintéticas são realistas? Quais as diferenças entre as sintéticas e as reais?
%   Os módulos das redes sintéticas também podem ser decompostos em módulos, como nas redes reais?
%
% O QUE MAIS
%
%   Ver no wiki as observações subjetivas sobre os modelos.
%   Começar pelo sistema pequeno, pra obter resultados mais rápidos, e então reproduzir o método com o sistema grande
%   A implementação do modelo de Rodrigo2008 precisa ser revista pra ficar mais eficiente (está muito lento!)
%   Podemos usar uma distância de Garcia usando o coeficiente de clustering em vez de usar a distância propriamente dita.


\begin{document}
\title{Artigo 1}
\author{Rodrigo Rocha Gomes e Souza}
\maketitle

\begin{abstract}

% goals, results, and the main conclusions of your study

% Kent Beck's sentences: The first states the problem. The second states why the problem is a problem. The third is my startling sentence. The fourth states the implication of my startling sentence.

\end{abstract}

\section{Introdução} % why

\cite{Newman2003}

% why you have investigated the question
% 
% how it relates to earlier research that has been done in the field
% 
% 1. Open with two or three sentences placing your study subject in context
% 2. Follow with a description of the problem and its history, including previous research
% 3. Describe how your work addresses a gap in existing knowledge or ability (here's where you'll state why you've undertaken this study). 
% 4. State what information your article will address. 

É difícil obter muitos dados, grandes sistemas. Difícil achar sistema com arquitetura conhecida.

embora existam alguns trabalhos avaliando algoritmos de recuperação de arquitetura, os resultados experimentais não dão pistas sobre por que os algoritmos são bons em uns critérios e ruins em outros, quais são as coisas que influenciam o desempenho do algoritmo.


\section{Métodos e Materiais} % how

O modelo de Lancichinetti não foi feito baseado em nenhum domínio em particular.

Programa usado para extrair dependências: DepFind. Análise estática. Quais interações foram consideradas. Lifting.

%particular techniques used and why, if relevant
%modifications of any techniques; be sure to describe the modification
%assumptions underlying the study 
%statistical methods, including software programs 

\section{Resultados} % what was found

Produção de um modelo, baseado em um modelo existente, de geração de redes de software com estrutura de módulos embutida.

Comparação desse modelo com um modelo presente na literatura usando como critério a semelhança com redes de software.

Análise desse modelo modelos de geração de redes com estrutura de módulos embutida, e da semelhança dessas redes com redes de software.

%  * present results clearly and logically
%  * avoid excess verbiage
%  * consider providing a one-sentence summary at the beginning of each paragraph if you think it will help your reader understand your data 

\section{Discussão} % why it's significant

O quão bem o modelo funciona.

Especulação sobre o papel da modelagem estatística na engenharia de software.

Focar na hipótese: redes sintéticos são uma boa aproximação de redes reais.

%how useful this technique is: how well did it work, what are the benefits and drawbacks, etc

%This section centers on speculation

%Focus your discussion around a particular question or hypothesis

Trabalhos futuros: explorar métricas de arquitetura, avaliar algoritmos de clustering, considerar outros modelos.
incluir pesos das arestas nas análises e nos modelos.

\bibliographystyle{apalike}
\bibliography{complex-networks,rodrigo-mestrado}

\end{document}
