\documentclass{acm_proc_article-sp}
\usepackage[utf8]{inputenc}
\usepackage[T1]{fontenc}
\usepackage[brazil]{babel}
\usepackage{graphicx}
\usepackage{url}

% Ordem de escrita
%   Resultados
%   Discussão
%   Introdução
%   Materiais e Métodos
%   Abstract
%   Título


% == PLANEJAMENTO DO EXPERIMENTO ==
% 
% INGREDIENTES
%
%   Dois sistemas, um pequeno (?) e um maior (IRPF)
%   Modelos de geração de redes com estrutura de módulos embutida: Rodrigo2008 e Lancichinetti
%   Modelos sem estrutura de módulos: barabasi, bollobas, configuration model.
%
% MODO DE PREPARO
%
%   Analisar métricas dos sistemas (distribuição de graus, distribuição do coeficiente de clustering, correlação de graus).
%   Extrair a arquitetura dos sistemas: quais são os módulos e quais módulos se relacionam. Analisar a distribuição dos tamanhos dos módulos.
%   "Tunar" os parâmetros dos modelos para gerar redes com as mesmas métricas dos sistemas reais
%   Rodar os modelos diversas vezes com os parâmetros encontrados para gerar diversas redes para cada sistema.
%   Comparar redes sintéticas com as redes reais correspondentes através da distância entre redes de Garcia.
% 
% RESULTADOS
%
%   Modelos com estrutura de módulos embutida resultam em redes mais parecidas com as redes reais do que os modelos sem módulos? (Em outras palavras: a informação sobre tamanhos dos módulos e a maneira como os módulos se ligam realmente são uma vantagem?)
%   As redes sintéticas são realistas? Quais as diferenças entre as sintéticas e as reais?
%   Os módulos das redes sintéticas também podem ser decompostos em módulos, como nas redes reais?
%
% O QUE MAIS
%
%   Ver no wiki as observações subjetivas sobre os modelos.
%   Começar pelo sistema pequeno, pra obter resultados mais rápidos, e então reproduzir o método com o sistema grande
%   A implementação do modelo de Rodrigo2008 precisa ser revista pra ficar mais eficiente (está muito lento!)
%   Podemos usar uma distância de Garcia usando o coeficiente de clustering em vez de usar a distância propriamente dita.


\begin{document}
\title{Artigo 1} % Avaliação de modelos de síntese de software
\author{Rodrigo Rocha Gomes e Souza}
\maketitle

\begin{abstract}

% goals, results, and the main conclusions of your study

% Kent Beck's sentences: The first states the problem. The second states why the problem is a problem. The third is my startling sentence. The fourth states the implication of my startling sentence.

\end{abstract}

\section{Introdução} % why

% why you have investigated the question
% 
% how it relates to earlier research that has been done in the field
% 
% 1. Open with two or three sentences placing your study subject in context
% 2. Follow with a description of the problem and its history, including previous research
% 3. Describe how your work addresses a gap in existing knowledge or ability (here's where you'll state why you've undertaken this study). 
% 4. State what information your article will address. 

Recuperação de arquitetura de software é o ato de extrair aspectos da arquitetura de um sistema através de artefatos como código-fonte. Muitos esforços têm se concentrado no uso de algoritmos de agrupamento (\emph{clustering}) com a finalidade de detectar módulos arquiteturais --- grupos coesos de entidades de código-fonte (classes ou funções).

Uma das formas de avaliar um algoritmo de agrupamento no contexto da recuperação de arquitetura consiste em aplicar o algoritmo a um sistema cuja estrutura de módulos seja conhecida. Infelizmente não existem muitos sistemas cuja estrutura de módulos é bem documentada. OS ESTUDOS FEITOS ASSIM NÃO SÃO CONTUNDENTES, OBTÊM RESULTADOS DIFERENTES PARA SISTEMAS DIFERENTES E NÃO TÊM NENHUMA PISTA SOBRE O QUE FAZ UM ALGORITMO SER BOM EM UM SISTEMA E RUIM EM OUTRO.

Propomos sistemas de software gerados por computador, COM ESTRUTURA DE MÓDULOS EMBUTIDA. VANTAGENS: É POSSÍVEL AJUSTAR PARÂMETROS DOS SISTEMAS GERADOS E ASSIM GANHAR INSIGHT SOBRE OS PARÂMETROS QUE INFLUENCIAM A ACURÁCIA DE UM ALGORITMO. EM PARTICULAR, É POSSÍVEL CONTROLAR O TAMANHO DO SOFTWARE GERADO E, ASSIM, GERAR SISTEMAS GRANDES, MAIS PARECIDOS COM OS QUE SERIAM ANALISADOS PELAS TÉCNICAS EM UM CENÁRIO REAL.

O modelo é comparado com outros modelos disponíveis na literatura sobre redes complexas.

%embora existam alguns trabalhos avaliando algoritmos de recuperação de arquitetura, os resultados experimentais não dão pistas sobre por que os algoritmos são bons em uns critérios e ruins em outros, quais são as coisas que influenciam o desempenho do algoritmo.

\section{TEORIA}

Neste estudo nos concentraremos em sistemas orientados a objeto, para simplificar, mas os conceitos provavelmente podem ser aplicados a outros paradigmas como procedimental ou funcional. 
Muitas técnicas de recuperação de arquitetura analisam uma representação abstrata de sistemas de software, as redes de dependências entre classes. Nessas redes, os vértices representam classes e existe uma aresta do vértice A para o vértice B se a classe A depende da classe B para funcionar corretamente. Essa dependência pode ser resultado diversos tipos de interação entre as classes: A estende B, um método de A chama um método de B etc.
A extração da rede de um sistema de software se dá através da análise estática de seu código-fonte ou do código objeto.

Neste estudo consideramos dois sistemas de software reais e três modelos de geração de redes complexas.

Neste estudo foram avaliados três modelos de geração de redes complexas: o modelo de configuração, o modelo de Bollobás \cite{Bollobas2003}, o modelo de Lancichinetti, Fortunato e Radicchi \cite{Lancichinetti2008} e um novo modelo, baseado no modelo de Bollobás. Os dois últimos modelos geram redes com uma estrutura de módulos embutida.

DESCRIÇÃO DE CADA MODELO

Bollobas: pensando na Web. Não tem estrutura de módulos embutida. Parâmetros são probabilidades.

O modelo de Lancichinetti não foi feito baseado em nenhum domínio em particular. Parâmetros são métricas da rede que se quer obter.

\section{O modelo novo}

Valorizar o modelo novo. Em relação a Bollobas, ele tem módulos embutidos. Em relação a Lancichinetti, pode especificar a arquitetura, mixing é variável, o grafo da arquitetura não é (necessariamente) completo.

\section{Experimento} % how

O estudo experimental foi dividido em cinco etapas: extração das redes de software reais, análise de sistemas reais, sintonia dos parâmetros dos modelos, geração de sistemas sintéticos e comparação entre redes sintéticas e redes reais. 

\subsection{Extração}

Os sistemas analisados são escritos em Java e distribuídos em diversos arquivos JAR. Alguns arquivos JAR representam código do próprio sistema e outros são bibliotecas usadas por eles. A ferramenta DepFind\footnote{\url{http://depfind.sourceforge.net/}} foi usada para extrair (estaticamente) todas as interações entre classes, métodos e atributos dos sistemas analisados. Foi considerado o sistema + as bibliotecas. A seguir um programa feito em casa abstraiu para dependências classe para classe (lifting). Consideramos que cada JAR é um módulo arquitetural.

\subsection{Análise de sistemas reais}

A rede resultante foi submetida a diversas análises em que foram coletadas métricas da teoria dos grafos e da teoria das redes complexas:

número de classes
expoente da distribuição de graus
número de módulos
expoente da distribuição dos tamanhos dos módulos
...

Para o ajuste do expoente ..., usamos maximum likelihood estimation etc. \cite{Clauset2007}. Usamos a implementação disponível em X\footnote{\url{http://www.santafe.edu/~aaronc/powerlaws/}}.

\subsection{Sintonia dos parâmetros dos modelos}

Distribuição de graus foi usada para todos os modelos.

Modelo de Lancichinetti tem os seguintes parâmetros.

No modelo de Bollobás os parâmetros não correspondem a métricas da rede resultante, e por isso foi preciso sintonizar os parâmetros de forma experimental (correspondência analítica existe, mas não consegui usar).

\subsection{Geração de sistemas sintéticos}

Foram gerados 10 redes para cada modelo. Por que 10?

\subsection{Comparação com os sistemas reais}

Usamos a métrica de distância entre redes definida por \cite{Andrade2008}, implementação de Charles.

Consideramos para cada modelo a média entre as distâncias de cada rede gerada pelo modelo.

%particular techniques used and why, if relevant
%modifications of any techniques; be sure to describe the modification
%assumptions underlying the study 
%statistical methods, including software programs 

\section{Resultados} % what was found

TABELA 1: sistemas analisados. Nome, versão, métricas (tamanho, número de módulos, ....)

TABELA 2 (ou GRÁFICO): distâncias.

%  * present results clearly and logically
%  * avoid excess verbiage
%  * consider providing a one-sentence summary at the beginning of each paragraph if you think it will help your reader understand your data 

\section{Discussão} % why it's significant

% Recapitulando o que este trabalho fez:

% Produção de um modelo, baseado em um modelo existente, de geração de redes de software com estrutura de módulos embutida.

% Comparação desse modelo com um modelo presente na literatura usando como critério a semelhança com redes de software.

% Análise desse modelo modelos de geração de redes com estrutura de módulos embutida, e da semelhança dessas redes com redes de software.

% -----

O quão bem o modelo funciona, vantagens e desvantagens em relação a outras abordagens (complementares).

Especulação sobre o papel da modelagem estatística na engenharia de software.

Focar na hipótese: redes sintéticas com parâmetros ajustáveis dão insights sobre as ferramentas de engenharia reversa / evolução de software.
% redes sintéticos são uma boa aproximação de redes reais

%how useful this technique is: how well did it work, what are the benefits and drawbacks, etc

%This section centers on speculation

%Focus your discussion around a particular question or hypothesis

Trabalhos futuros: explorar métricas de arquitetura, avaliar algoritmos de clustering, considerar outros modelos.
incluir pesos das arestas nas análises e nos modelos.

%\bibliographystyle{apalike}
\bibliographystyle{abbrv}
\bibliography{complex-networks,rodrigo-mestrado}

\end{document}
