% Li em algum lugar que é bom citar muitas pessoas nos agradecimentos de um livro, porque assim muitas pessoas compram-no para ver seu nome escrito.
Muito obrigado à minha família, que me ajudou de inúmeras maneiras em todos os momentos deste empreendimento acadêmico: meu pai, Renato, minha mãe, Ligia, e meu irmão, Henrique; minha esposa, Denise, meu sogro, Ângelo, e minha sogra, Fátima.

Obrigado à minha orientadora de graduação, Christina, que me apoiou antes, durante e depois do meu mestrado, de incontáveis formas.

Obrigado a Maria da Guia e Oscar, por nos acolherem (a mim e a Denise) sobretudo na nossa primeira visita a Campina Grande.

Obrigado aos meus orientadores, Dalton e Jorge, que me propuseram como tema de mestrado um desafio empolgante, e que sempre me motivaram a fazer o meu melhor.

Obrigado ao pessoal do Grupo de Métodos Formais (GMF) da UFCG, por servirem de inspiração para o meu trabalho, em especial a Roberto, a quem considero meu co-orientador. Obrigado aos colegas da UFCG, pela companhia, em especial a Dalton Cézane, Isabel, Stéfani e Rafael. Muito obrigado a Anne Caroline e a Bruno, pela grande amizade. Obrigado a Lilian do GMF e a Aninha da COPIN, pela atenção.

Obrigado a Lemuel e aos colegas do Coro em Canto, responsáveis por muitas das boas lembranças que tenho de Campina Grande. Obrigado a Marta pela amizade e ajuda.

Obrigado à CAPES, pelo apoio financeiro no primeiro ano, sob a forma de bolsa de mestrado. Obrigado à Fundação de Amparo à Pesquisa do Estado da Bahia (Fapesb) --- em especial, Sandra (coordenadora de TI) e Dora (então diretora geral) ---, pelo apoio financeiro no segundo ano do mestrado, através de um emprego flexível como analista de sistemas na Coordenação de TI. Obrigado aos colegas da Fapesb pela convivência agradável. Obrigado ao Doutorado Multi-Institucional em Ciência da Computação (DMCC) e à Fapesb, pelo apoio financeiro nos últimos seis meses, desta vez na forma de uma bolsa de doutorado. Obrigado à Apple, pelo apoio financeiro nos últimos seis meses, através da venda dos aplicativos da RoDen Apps para iOS (iPhone, iPod touch e iPad).

Obrigado a Fabíola e a Ítalo, que me permitiram rodar os experimentos do mestrado no \emph{cluster} do Grupo de Algoritmos e Computação Distribuída (Gaudi) da UFBA.

Obrigado a todos os que acompanharam e opinaram sobre meu trabalho em pelo menos um momento: os professores do Laboratório de Engenharia de Software (LES) da UFBA; Charles e Garcia, do grupo de Física Estatística e Sistemas Complexos (FESC) da UFBA; a professora Rose, do Departamento de Estatística da UFBA; Lancichinetti e Gail Murphy, pelo \emph{feedback} via e-mail, e Nenad Medvidović, pelo \emph{feedback} nos momentos finais; e Fubica, Eustáquio, Raquel e Marco Túlio, que participaram de bancas de avaliação ao longo do desenvolvimento deste trabalho e contribuíram com observações valiosas.

Obrigado a Deus por ter colocado tantas pessoas maravilhosas no meu caminho.