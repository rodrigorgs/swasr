% (15-25)
% 
%     * redes complexas
%     * designs de software como redes complexas
%     * conceito de tríadas como base para caracterização de redes complexas
%     * clustering
%     * avaliação de algoritmos de clustering

\chapter{Agrupamento de Software}

\begin{section}{Algoritmos de agrupamento}

Mineração de dados.

Na física, detecção de comunidades.

Critérios: validade interna, validade externa etc.

\end{section}
\begin{section}{Agrupamento de software}

De um ponto de vista amplo, agrupa artefatos de software de um sistema computacional. Código-fonte, UML, documentação...

Uma vertente, estudada neste trabalho, é o agrupamento de entidades do código-fonte. O próprio conceito de entidade não é bem definido. Podem ser funções e variáveis em sistema procedimentais CITE, tipos abstratos de dados CITE, arquivos-fonte CITE, classes em um sistema orientado a objetos CITE etc.

Há vários critérios: nomes de identificadores CITE, co-mudança em sistemas de controle de versão CITE, co-uso em tempo de execução CITE, relações estáticas no código-fonte etc.

Neste trabalho estudaremos relações estáticas em programas orientados a objetos. Ainda assim, um software pode ser representado de diversas formas.

\end{section}
\begin{section}{Grafo de dependências entre entidades}

Grafo orientado não-ponderado.
Entidades: classes e interfaces Java.

Significado de ``dependência''.

Limitações da análise estática.

Ferramenta DependencyFinder.

Dependências:
  classe é subclasse de classe
  classe implementa interface
  classe chama método de classe
  classe acessa atributo de classe
  ...

\end{section}
\begin{section}{Algoritmos de agrupamento usados em software e selecionados para este trabalho}

Hierárquico (mineração de dados), usado por Anquetil, Wu etc.

ACDC (ES)

Bunch (ES)

Infomap (física estatística, detecção de comunidades), estudado por Lancichinetti.

\end{section}

% Avaliação de agrupamento de software deve ser uma seção separada ou parte da introdução e da seção de agrupamento em mineração de dados?

