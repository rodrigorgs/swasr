\usepackage[brazilian,portuges,english]{babel}
% \usepackage{ae} 
\usepackage[utf8x]{inputenc}
\usepackage[T1]{fontenc} 

%\usepackage[portuges,english]{babel}
\usepackage{copin,mestre} % ,epsfig
\usepackage{times}
\usepackage{multirow}
\usepackage{ucs}

%-----------------------------------------------------------------------------------------------------

\usepackage{url}
\usepackage{fancyheadings}
\usepackage{graphicx}
\usepackage{longtable} %tabelas longas, que ultrapassam uma pagina

\usepackage{listings}
\lstset{numbers=left,
stepnumber=1,
firstnumber=1,
%numberstyle=\tiny,
extendedchars=true,
breaklines=true,
frame=tb,
basicstyle=\footnotesize,
stringstyle=\ttfamily,
showstringspaces=false
}
\renewcommand{\lstlistingname}{C\'odigo Fonte}
\renewcommand{\lstlistlistingname}{Lista de C\'odigos Fonte}

\selectlanguage{portuges}
%\selectlanguage{brazilian}
\sloppy

\begin{document}

%%%%%%%%%%%%%%%%%%%%%%%%%%%%%%%%%%%%%%%%%%%%%%%%%%%%%%%%%%%%%%%%%%%%%%%%%%%%%%%%

\Titulo{Modelos Realistas para a Avaliação de Algoritmos de Agrupamento de Software}
\Autor{Rodrigo Rocha Gomes e Souza}
\Data{01/04/2010}
\Area{Ciência da Computação}
\Pesquisa{Engenharia de Software}
\Orientadores{Dalton Dario Serey Guerrero  
	 (Orientador) \\
	Jorge César Abrantes de Figueiredo 
	(Orientador)
	}

\newpage
\cleardoublepage

\PaginadeRosto

\newpage
\cleardoublepage

%%%%%%%%%%%%%%%%%%%%%%%%%%%%%%%%%%%%%%%%%%%%%%%%%%%%%%%%%%%%%%%%%%%%%%%%%%%%%%
\begin{resumo} 
Algoritmos de agrupamento de software agrupam em módulos as entidades do código-fonte de sistemas de software (por exemplo, classes, funções etc.), facilitando a documentação da arquitetura de sistemas. A avaliação empírica desses algoritmos, no entanto, é dificultada pelo fato de existirem poucos sistemas com agrupamentos de referência para serem comparados com os agrupamentos encontrados pelos algoritmos. Neste trabalho é proposta uma abordagem de avaliação usando modelos que produzem representações de sistemas de software organizados em módulos. A abordagem é validada através de um experimento que mostra que os modelos estudados produzem grafos que se assemelham ao grafo de dependências entre entidades de sistemas de software típicos. XXX

\end{resumo}

\newpage
\cleardoublepage

%%%%%%%%%%%%%%%%%%%%%%%%%%%%%%%%%%%%%%%%%%%%%%%%%%%%%%%%%%%%%%%%%%%%%%%%%%%%%%
\begin{summary}
% Software clustering algorithms group source code entities into modules, facilitating software architecture documentation. Empirical evaluation of the algorithms, however, is difficult because there are few software systems with reference clusterings for comparison with the clusterings found by algorithms. In this thesis we propose an approach based on models that generate graphs representing software systems with built-in reference clusterings. The approach is validated through an experiment that shows that the models produce graphs resembling the graph of static dependencies between classes in object oriented software systems. Finally, clustering algorithms are evaluated through model generated graphs. It is expected that this study will increase the available knowledge about clustering algorithms, which can contribute to their improvement in the future.

The analysis of dependencies between source code entities of a software system is performed by several reverse engineering tools in order to reveal information that is useful for software maintenance. There is, however, a shortage of experimental studies designed to evaluate such tools, in part due to the high cost of conducting experiments in the area.

In the area of networks and distributed systems, the high cost of experimentation motivates the use of simulation as a means to evaluate protocols and algorithms. In reverse engineering, however, simulations are underexplored --- which is partly explained by the lack of realistic computational models for dependencies between source code entities.

This work presents three dependency models in order to support the evaluation of reverse engineering tools via controlled simulations. One of them, called BCR+, was developped in the context of this work. An evaluation of the models showed that, with an appropriate choice of parameters, they produce dependency networks that are structurally similar to dependencies extracted from real software systems. Moreover, classification rules were derived to predict, with 80\% accuracy, whether a network that was generated with certain parameter values will be similar to networks extracted from software systems.

This work also presents a proof of concept, demonstrating the feasibility of using one of the models to evaluate algorithms used in the context of software architecture recovery, a branch of reverse engineering.

\end{summary}

\newpage
\cleardoublepage

%%%%%%%%%%%%%%%%%%%%%%%%%%%%%%%%%%%%%%%%%%%%%%%%%%%%%%%%%%%%%%%%%%%%%%%%%%%%%%
\begin{agradecimentos}
% Li em algum lugar que é bom citar muitas pessoas nos agradecimentos de um livro, porque assim muitas pessoas compram-no para ver seu nome escrito.
Muito obrigado à minha família, que me ajudou de inúmeras maneiras em todos os momentos deste empreendimento acadêmico: meu pai, Renato, minha mãe, Ligia, e meu irmão, Henrique; minha esposa, Denise, meu sogro, Ângelo, e minha sogra, Fátima.

Obrigado à minha orientadora de graduação, Christina, que me apoiou antes, durante e depois do meu mestrado, de incontáveis formas.

Obrigado a Maria da Guia e Oscar, por nos acolher (a mim e a Denise) sobretudo na nossa primeira visita a Campina Grande.

Obrigado aos meus orientadores, Dalton e Jorge, que me propuseram como tema de mestrado um desafio empolgante, e que sempre me motivaram a fazer o meu melhor.

Obrigado ao pessoal do Grupo de Métodos Formais (GMF) da UFCG, por servirem de inspiração para o meu trabalho, em especial a Roberto, a quem considero meu co-orientador. Obrigado aos colegas da UFCG, pela companhia, em especial a Dalton Cézane, Isabel, Stéfani e Rafael. Muito obrigado a Anne Caroline e a Bruno, pela grande amizade. Obrigado a Lilian do GMF e a Aninha da COPIN, pela atenção.

Obrigado a Lemuel e aos colegas do Coro em Canto, responsáveis por muitas das boas lembranças em Campina Grande. Obrigado a Marta pela amizade e ajuda.

Obrigado à CAPES, pelo apoio financeiro no primeiro ano, sob a forma de bolsa de mestrado. Obrigado à Fundação de Amparo à Pesquisa do Estado da Bahia (Fapesb) --- em especial, Sandra (coordenadora de TI) e Dora (então diretora geral) ---, pelo apoio financeiro no segundo ano do mestrado, através de um emprego flexível como analista de sistemas na Coordenação de TI. Obrigado aos colegas da Fapesb pela convivência agradável. Obrigado ao Doutorado Multi-Institucional em Ciência da Computação (DMCC) e à Fapesb, pelo apoio financeiro nos últimos seis meses, desta vez na forma de uma bolsa de doutorado. Obrigado à Apple, pelo apoio financeiro nos últimos seis meses, através da venda dos aplicativos da RoDen Apps para iOS (iPhone, iPod touch e iPad).

Obrigado a Fabíola e a Ítalo, que me permitiram rodar os experimentos do mestrado no \emph{cluster} do Grupo de Algoritmos e Computação Distribuída (Gaudi) da UFBA.

Obrigado a todos os que acompanharam e opinaram sobre meu trabalho em pelo menos um momento: os professores do Laboratório de Engenharia de Software (LES); Charles e Garcia, do grupo de Física Estatística e Sistemas Complexos (FESC) da UFBA; a professora Rose, do Departamento de Estatística da UFBA; Lancichinetti, Gail Murphy e Nenad Medvidović; e Fubica, Eustáquio, Raquel e Marco Túlio, que participaram de bancas de avaliação ao longo do desenvolvimento deste trabalho e contribuíram com observações valiosas.

Obrigado a Deus por ter colocado tantas pessoas maravilhosas no meu caminho.
\end{agradecimentos}

\clearpage

%%%%%%%%%%%%%%%%%%%%%%%%%%%%%%%%%%%%%%%%%%%%%%%%%%%%%%%%%%%%%%%%%%%%%%%%%%%%%%

%% Definicao do cabecalho: secao do lado esquerdo e numero da pagina do lado direito
\pagestyle{fancy}
\addtolength{\headwidth}{\marginparsep}\addtolength{\headwidth}{\marginparwidth}\headwidth = \textwidth
\renewcommand{\chaptermark}[1]{\markboth{#1}{}}
\renewcommand{\sectionmark}[1]{\markright{\thesection\ #1}}\lhead[\fancyplain{}{\bfseries\thepage}]%
	     {\fancyplain{}{\emph{\rightmark}}}\rhead[\fancyplain{}{\bfseries\leftmark}]%
             {\fancyplain{}{\bfseries\thepage}}\cfoot{}

%%%%%%%%%%%%%%%%%%%%%%%%%%%%%%%%%%%%%%%%%%%%%%%%%%%%%%%%%%%%%%%%%%%%%%%%%%%%%%%%
\selectlanguage{portuges}

\Sumario
\ListadeSimbolos
\listoffigures
\listoftables
\lstlistoflistings %lista de codigos fonte - Para inserir a listagem de codigos fonte
\newpage
\cleardoublepage

\Introducao


%%%%%%%%%%%%%%%%%%%%%%%%%%%%%%%%%%%%%%%%%%%%%%%%%%%%%%%%%%%%%%%%%%%%%%%%%%%%%%%%
%
% Hifenizacao - Colocar lista de palavras que nao devem ser separadas e que 
% nao estao no dicionario portugues.
% As palavras do dicionario portugues ja sao separadas corretamente pelo lateX
%
\hyphenation{hardware software}
