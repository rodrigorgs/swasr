\chapter{Um Classificador de Redes}

% Introdução
% Definição do Classificador
% Avaliação do Classificador
% - acurácia de teste: prevê a acurácia de execução
% Indução de um Modelo de Classificação
% - acurácia de treinamento

\begin{section}{Introdução}
	Os três modelos de redes apresentados anteriormente --- BCR+, LFR e CGW --- geram redes organizadas em módulos. Tal condição, no entanto, não é suficiente para que os modelos sejam usados em estudos sobre ferramentas de engenharia reversa. É importante que os modelos gerem redes que se assemelham a redes extraídas de sistemas de software reais.
	
	Daqui para frente, redes que se assemelham a redes de software serão chamadas de \emph{redes software-realistas}. Redes de software são, por definição, software-realistas. Redes estudadas em outros domínios (ex.: redes biológicas) são, por pressuposto, não software-realistas.
	
	Para avaliar se os modelos de redes geram redes software-realistas, é preciso haver um \emph{modelo de classificação} que, dada uma rede, determine se ela é ou não software-realista a partir da análise da estrutura da rede. De forma geral, o problema de classificação consiste em atribuir objetos a classes pré-determinadas a partir da análise de atributos dos objetos. Nesta pesquisa, buscamos modelos de classificação que determinam se uma rede (objeto) é software-realista ou não-software realista (classes).
\end{section}

\begin{section}{Visão Geral}
	O problema de classificação é estudado na disciplina de aprendizagem de máquina. Nessa disciplina, o problema de classificação é resolvido a partir da criação de um classificador. Um \emph{classificador} é um algoritmo que induz um modelo de classificação a partir de um conjunto que contém exemplos de objetos pertencentes a todas as classes relevantes, denominado \emph{conjunto de treinamento}. 

	Um modelo de classificação determina a classe de um objeto com base apenas na análise de seus atributos. Um modelo de classificação é induzido de forma que apresente alta taxa de acerto quando aplicada ao conjunto de treinamento. Mais importante, no entanto, é que o modelo apresente alta taxa de acerto quando aplicado a objetos que não foram usados em seu treinamento. Nesse caso, diz-se que o modelo de classificação é generalizável.
	
	Para avaliar se um modelo é generalizável, é selecionado um conjunto de teste, contendo objetos cujas classes são conhecidas e que não foram usados no treinamento. O modelo induzido a partir do conjunto de treinamento é então aplicado ao conjunto de teste, e a taxa de acerto é calculada comparando-se as classes reais dos objetos do conjunto de teste com as classes determinadas pelo modelo.
	
	Para avaliar se um classificador induz modelos de classificação generalizáveis, o classificador é executado diversas vezes, variando-se o conjunto de treinamento e o conjunto de teste, e então é calculada a taxa de acerto do modelo induzido quando aplicado ao conjunto de teste. Desta forma é possível ter uma estimativa das taxas de acertos dos modelos induzidos pelo classificador.
	
\end{section}	
	
\begin{section}{Definição do Classificador}
	
	Propomos modelos de classificação da forma m(x, R, S0), onde $x$ é a rede a ser classificada, $R$ é um conjunto de redes software-realistas, e S é um número real entre -1,0 e 1,0:
	
\begin{verbatim}
	m(x, R, S0) =
	software-realista, se S(x, R) > S0
	não software-realista, caso contrário
\end{verbatim}
	
	A função $\mathrm{S}(x, R)$ representa o grau de software-realismo de uma rede. O valor de $S_0$ é chamado de limiar de software-realismo. Quando o grau de software-realismo da rede $x$ supera o limiar, a rede é classificada como software-realista; nos demais casos, a rede é classificada como não software-realista.
	
	A definição de $\mathrm{S}(x, R)$ se baseia métrica de similaridade entre redes, sim($x$, $y$), definida no Capítulo XXX como o coeficiente de correlação entre os perfis de concentração de tríades das redes. O valor da função $\mathrm{S}(x, R)$ é definido como a média aritmética dos valores de similaridade entre $x$ e as redes de $R$:

	$$
	\mathrm{S}(x, R) ~=~ \frac{
	\displaystyle\sum_{s \in R} \mathrm{sim}(x, s)
	}{|R|} \mathrm{,}
	$$

	O valor de $\mathrm{S}(x, R)$ varia entre -1,0 e 1,0. Quanto maior o valor, maior o grau de software-realismo.
	
	O valor do limiar afeta diretamente a taxa de acerto...

	Um modelo de classificação da forma m(x, R, S0) pode ser induzido por um classificador apresentado a seguir.

	---

	Um classificador recebe um conjunto de treinamento e retorna um modelo de classificação da forma m(x, R, S0). O conjunto de treinamento pode ser dividido em dois conjuntos: o conjunto T, contendo apenas redes software-realistas, e o conjunto ~T, contendo apenas redes não-software realistas. Os conjuntos T e ~T fornecem uma definição baseada em exemplos do conceito de software-realismo.
	
	Para induzir um modelo da forma m(x, R, S0), é preciso determinar R e S0. O conjunto R é determinado como as redes software-realistas do conjunto de treinamento, isto é, R = T. O valor de S0 é determinado através de um algoritmo de aprendizagem que analisa os conjuntos T e ~T.
	
	\begin{verbatim}
	função calcula_s0
	=================
	Seja maior_acurácia = 0
	Seja limiar = 0
	Para cada rede x em (R U ~R), faça
	  Seja s = S(x, R)  # candidato a limiar
	  # calcula o acurácia quando S_0 = s
		Seja c = 0
	  Para cada rede y em (R U ~R), faça
	    Se (S(y, R) >= s AND y in R)
	       OR (S(y, R) < s AND x in ~R)), faça
	      c = c + 1
	    Fim-se
	  Fim-para
	  acurácia = c / |R U ~R|
	  # verifica se é o maior acurácia já encontrado
	  Se acurácia > maior_acurácia
	    maior_acurácia = acurácia
	    limiar = s
	  Fim-se
	Retorne limiar
	\end{verbatim}
	 
Em poucas palavras, o algoritmo 

\end{section}

\begin{section}{rascunho}
		O modelo deve não apenas representar bem o conjunto de treinamento, mas também ser capaz de determinar corretamente a classe de objetos que não foram usados no seu treinamento. Nesse caso, diz-se que o modelo de classificação é generalizável.
		
		
		
		generalizável
		
		Nesse conjunto, a classe de cada objeto é conhecida. O modelo de classificação induzido relaciona atributos dos objetos às classes de forma a maximizar a taxa de acerto.
		
		
		
		taxa de acerto
		
		Nesta seção, é apresentado um classificador que induz modelos de classificação de redes do ponto de vista do software-realismo.
		
		Nessa disciplina, modelos de classificação são induzidos por um classificador a partir de um conjunto que contém exemplos de objetos de todas as categorias. Um classificador nada mais é do que 
		
		%, que se preocupa com o desenvolvimento de métodos para a construção de modelos a partir de exemplos. 
		
		developing methods for software to learn from experience or extract knowledge from examples in a database.
		
		
		% (Problema da Pesquisa. Mapeamento para Mineração de Dados. Revisão de Mineração de Dados. Execução. Resultados.)
		% Buscamos um modelo de classificação de redes, que determina se uma rede é ou não software-realista.
		% 
		% (Definir software-realista)
		% 
		% Mineração de dados é o processo de automaticamente descobrir informação útil em grandes repositórios de dados.
		% 
		% Modelos de classificação são tema de estudo da área de mineração de dados. Classificação: atribuir objetos a categorias pré-determinadas. Nosso caso: objetos = redes; categorias = software-realista, não software-realista.
		% 
		% Na abordagem de mineração de dados, os modelos são induzidos a partir de exemplos de objetos pertencentes a todas as categorias de interesse.
	
\end{section}
