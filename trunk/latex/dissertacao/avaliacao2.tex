\chapter{Avaliação de Modelos de Redes} \label{cap:avaliacao}

\begin{section}{Introdução}
Os três modelos apresentados anteriormente --- BCR+, LFR e CGW --- geram redes organizadas em módulos. Tal condição, no entanto, não é suficiente para que os modelos sejam usados em estudos sobre ferramentas de engenharia reversa. É importante que os modelos gerem redes que se assemelham a redes extraídas de sistemas de software reais.

Daqui para frente, redes que se assemelham a redes de software serão chamadas de \emph{redes software-realistas}. Redes de software são, por definição, software-realistas. Redes estudadas em outros domínios (ex.: redes biológicas) são, por pressuposto, não software-realistas.

O objetivo deste capítulo é avaliar se os modelos são capazes de gerar redes software-realistas. , com uma escolha adequada de valores para os parâmetros,

Para apoiar a avaliação dos modelos, foi desenvolvida uma função que classifica redes em duas categorias: redes software-realistas e redes não software-realistas. Essa função foi então aplicada a redes geradas pelos três modelos. Como resultado, constatou-se que todos os modelos geram tanto redes software-realistas quanto redes não software-realistas. A partir daí foram identificadas regras práticas que permitem prever a categoria de uma rede gerada pelos modelos a partir da análise dos parâmetros usados na geração.

% Este capítulo está divido em quatro seções. Na Seção XXX é apresentada uma função de classificação, C($x$), que define formalmente o conceito de software-realismo. A função classifica uma rede em software-realista ou não software-realista com base apenas na estrutura da rede. 
% 
% Na Seção XXX é relatado um experimento realizado com a finalidade de validar a função de classificação. Dois critérios são avaliados:
% 
% \begin{itemize}
% 	\item a função deve classificar como software-realista as redes que são extraídas a partir de sistemas de software;
% 	\item a função deve classificar como não software-realista as redes que são estudadas fora do domínio de software (ex.: redes biológicas, redes sociais, redes linguísticas).
% \end{itemize}
% 
% O segundo critério faz sentido porque pesquisas anteriores já mostraram que os domínios exemplificados possuem perfis de concentração de tríades distinguíveis entre si. É de se esperar que o mesmo ocorra com o domínio de software em relação aos demais domínios. De fato, o experimento revelou que a função de classificação escolhida atende aos dois critérios.
% 
% Na Seção XXX a função de classificação é aplicada a milhares de redes geradas pelos modelos. Como resultado, constatou-se que todos os modelos geram tanto redes software-realistas quanto redes não software-realistas. Além disso, foram identificadas regras práticas que relacionam, com alta acurácia, valores de parâmetros dos modelos com a classificação da rede resultante.
% 
% A Seção XXX fornece um sumário dos resultados.

\end{section}

