Introdução
Objetivo
Material e Métodos
Resultados
Conclusão

%%%%%%%%%%%%%%%%%%%%%%%%%%%%%%%%%%%%%%%%

    *  Introdução. O problema do desenvolvimento de grandes sistemas; técnicas de recuperação de arquitetura como parte da solução. Dificuldade de se avaliar essas técnicas. Proposta do trabalho: avaliação através de redes de software sintéticas. Objetivo: avaliar se redes sintéticas se assemelham às reais.
    * Conceitos. Recuperação de arquitetura. Redes de software. Redes complexas. Modelos de redes.
    * Trabalhos relacionados. Revisão crítica de trabalhos de avaliação empírica de técnicas de recuperação arquitetural.
    * Avaliação empírica de modelos de redes. Experimento que mede a similaridade entre redes sintetizadas pelos modelos e redes reais. Discussão dos métodos e resultados.
    * Estudo de caso. Aplicação da avaliação proposta para estudar técnicas pré-selecionadas de recuperação de arquitetura. Comparação dos resultados obtidos em trabalhos relacionados.
    * Conclusão. Recapitulação, contribuições e possíveis trabalhos futuros.
    * Referências. Referências bibliográficas.

%%%%%%%%%%%%%%%%%%%%%%%%%%%%%%%%%%%%%%%%

Resumo (inclui resultados)
Introdução
  Objetivos geral e específicos
  Resultados esperados
  Limitações do trabalho
  Métodos
  Justificativa
  Descrição dos demais capítulos
Revisão bibliográfica (Fundamentação teórica)
  Conceitos
  Trabalhos correlatos
  (Evolução das ideias)
Desenvolvimento
  Definições (constitutivas/operacionais)
  Métodos?
  Apresentação e análise dos dados
Conclusões (liga desenvolvimento a introdução)
  (Limitações)
  (Lições aprendidas)
  Recapitulação
  Resumo das contribuições
  Trabalhos futuros
Referências

