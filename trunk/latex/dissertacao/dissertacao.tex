\documentclass[a4paper,titlepage]{copin}
\usepackage[brazilian,portuges,english]{babel}
\usepackage[T1]{fontenc} 
\usepackage{ae} 
\usepackage[utf8]{inputenc}

%\usepackage[portuges,english]{babel}
\usepackage{copin,mestre,epsfig}
\usepackage{times}
\usepackage{multirow}
\usepackage{ucs}
%\usepackage[utf8]{inputenc}
%\usepackage[T1]{fontenc}
%-----------------------------------------------------------------------------------------------------

\usepackage{url}
\usepackage{fancyheadings}
\usepackage{graphicx}
\usepackage{longtable} %tabelas longas, que ultrapassam uma pagina

\usepackage{listings}
\lstset{numbers=left,
stepnumber=1,
firstnumber=1,
%numberstyle=\tiny,
extendedchars=true,
breaklines=true,
frame=tb,
basicstyle=\footnotesize,
stringstyle=\ttfamily,
showstringspaces=false
}
\renewcommand{\lstlistingname}{C\'odigo Fonte}
\renewcommand{\lstlistlistingname}{Lista de C\'odigos Fonte}

\selectlanguage{portuges}
%\selectlanguage{brazilian}
\sloppy

\begin{document}

%%%%%%%%%%%%%%%%%%%%%%%%%%%%%%%%%%%%%%%%%%%%%%%%%%%%%%%%%%%%%%%%%%%%%%%%%%%%%%%%
\Titulo{Redes software-realistas na avaliação de algoritmos de agrupamento de software}
\Autor{Rodrigo Rocha Gomes e Souza}
\Data{15/03/2010}
\Area{Ciência da Computação}
\Pesquisa{Engenharia de Software}
\Orientadores{Dalton Dario Serey Guerrero  
	 (Orientador) \\
	Jorge César Abrantes de Figueiredo 
	(Orientador)
	}

\newpage
\cleardoublepage

\PaginadeRosto

\newpage
\cleardoublepage

%%%%%%%%%%%%%%%%%%%%%%%%%%%%%%%%%%%%%%%%%%%%%%%%%%%%%%%%%%%%%%%%%%%%%%%%%%%%%%
\begin{resumo} 
Algoritmos de agrupamento de software agrupam em módulos as entidades do código-fonte de sistemas de software (por exemplo, classes, funções etc.), facilitando a documentação da arquitetura de sistemas. A avaliação empírica desses algoritmos, no entanto, é dificultada pelo fato de existirem poucos sistemas com agrupamentos de referência para serem comparados com os agrupamentos encontrados pelos algoritmos. Neste trabalho é proposta uma abordagem de avaliação usando modelos que produzem representações de sistemas de software organizados em módulos. A abordagem é validada através de um experimento que mostra que os modelos estudados produzem grafos que se assemelham ao grafo de dependências entre entidades de sistemas de software típicos. XXX

\end{resumo}

\newpage
\cleardoublepage

%%%%%%%%%%%%%%%%%%%%%%%%%%%%%%%%%%%%%%%%%%%%%%%%%%%%%%%%%%%%%%%%%%%%%%%%%%%%%%
\begin{summary}
% Software clustering algorithms group source code entities into modules, facilitating software architecture documentation. Empirical evaluation of the algorithms, however, is difficult because there are few software systems with reference clusterings for comparison with the clusterings found by algorithms. In this thesis we propose an approach based on models that generate graphs representing software systems with built-in reference clusterings. The approach is validated through an experiment that shows that the models produce graphs resembling the graph of static dependencies between classes in object oriented software systems. Finally, clustering algorithms are evaluated through model generated graphs. It is expected that this study will increase the available knowledge about clustering algorithms, which can contribute to their improvement in the future.

The analysis of dependencies between source code entities of a software system is performed by several reverse engineering tools in order to reveal information that is useful for software maintenance. There is, however, a shortage of experimental studies designed to evaluate such tools, in part due to the high cost of conducting experiments in the area.

In the area of networks and distributed systems, the high cost of experimentation motivates the use of simulation as a means to evaluate protocols and algorithms. In reverse engineering, however, simulations are underexplored --- which is partly explained by the lack of realistic computational models for dependencies between source code entities.

This work presents three dependency models in order to support the evaluation of reverse engineering tools via controlled simulations. One of them, called BCR+, was developped in the context of this work. An evaluation of the models showed that, with an appropriate choice of parameters, they produce dependency networks that are structurally similar to dependencies extracted from real software systems. Moreover, classification rules were derived to predict, with 80\% accuracy, whether a network that was generated with certain parameter values will be similar to networks extracted from software systems.

This work also presents a proof of concept, demonstrating the feasibility of using one of the models to evaluate algorithms used in the context of software architecture recovery, a branch of reverse engineering.

\end{summary}

\newpage
\cleardoublepage

%%%%%%%%%%%%%%%%%%%%%%%%%%%%%%%%%%%%%%%%%%%%%%%%%%%%%%%%%%%%%%%%%%%%%%%%%%%%%%
\begin{agradecimentos}
% Li em algum lugar que é bom citar muitas pessoas nos agradecimentos de um livro, porque assim muitas pessoas compram-no para ver seu nome escrito.
Muito obrigado à minha família, que me ajudou de inúmeras maneiras em todos os momentos deste empreendimento acadêmico: meu pai, Renato, minha mãe, Ligia, e meu irmão, Henrique; minha esposa, Denise, meu sogro, Ângelo, e minha sogra, Fátima.

Obrigado à minha orientadora de graduação, Christina, que me apoiou antes, durante e depois do meu mestrado, de incontáveis formas.

Obrigado a Maria da Guia e Oscar, por nos acolher (a mim e a Denise) sobretudo na nossa primeira visita a Campina Grande.

Obrigado aos meus orientadores, Dalton e Jorge, que me propuseram como tema de mestrado um desafio empolgante, e que sempre me motivaram a fazer o meu melhor.

Obrigado ao pessoal do Grupo de Métodos Formais (GMF) da UFCG, por servirem de inspiração para o meu trabalho, em especial a Roberto, a quem considero meu co-orientador. Obrigado aos colegas da UFCG, pela companhia, em especial a Dalton Cézane, Isabel, Stéfani e Rafael. Muito obrigado a Anne Caroline e a Bruno, pela grande amizade. Obrigado a Lilian do GMF e a Aninha da COPIN, pela atenção.

Obrigado a Lemuel e aos colegas do Coro em Canto, responsáveis por muitas das boas lembranças em Campina Grande. Obrigado a Marta pela amizade e ajuda.

Obrigado à CAPES, pelo apoio financeiro no primeiro ano, sob a forma de bolsa de mestrado. Obrigado à Fundação de Amparo à Pesquisa do Estado da Bahia (Fapesb) --- em especial, Sandra (coordenadora de TI) e Dora (então diretora geral) ---, pelo apoio financeiro no segundo ano do mestrado, através de um emprego flexível como analista de sistemas na Coordenação de TI. Obrigado aos colegas da Fapesb pela convivência agradável. Obrigado ao Doutorado Multi-Institucional em Ciência da Computação (DMCC) e à Fapesb, pelo apoio financeiro nos últimos seis meses, desta vez na forma de uma bolsa de doutorado. Obrigado à Apple, pelo apoio financeiro nos últimos seis meses, através da venda dos aplicativos da RoDen Apps para iOS (iPhone, iPod touch e iPad).

Obrigado a Fabíola e a Ítalo, que me permitiram rodar os experimentos do mestrado no \emph{cluster} do Grupo de Algoritmos e Computação Distribuída (Gaudi) da UFBA.

Obrigado a todos os que acompanharam e opinaram sobre meu trabalho em pelo menos um momento: os professores do Laboratório de Engenharia de Software (LES); Charles e Garcia, do grupo de Física Estatística e Sistemas Complexos (FESC) da UFBA; a professora Rose, do Departamento de Estatística da UFBA; Lancichinetti, Gail Murphy e Nenad Medvidović; e Fubica, Eustáquio, Raquel e Marco Túlio, que participaram de bancas de avaliação ao longo do desenvolvimento deste trabalho e contribuíram com observações valiosas.

Obrigado a Deus por ter colocado tantas pessoas maravilhosas no meu caminho.
\end{agradecimentos}

\clearpage

%%%%%%%%%%%%%%%%%%%%%%%%%%%%%%%%%%%%%%%%%%%%%%%%%%%%%%%%%%%%%%%%%%%%%%%%%%%%%%

%% Definicao do cabecalho: secao do lado esquerdo e numero da pagina do lado direito
\pagestyle{fancy}
\addtolength{\headwidth}{\marginparsep}\addtolength{\headwidth}{\marginparwidth}\headwidth = \textwidth
\renewcommand{\chaptermark}[1]{\markboth{#1}{}}
\renewcommand{\sectionmark}[1]{\markright{\thesection\ #1}}\lhead[\fancyplain{}{\bfseries\thepage}]%
	     {\fancyplain{}{\emph{\rightmark}}}\rhead[\fancyplain{}{\bfseries\leftmark}]%
             {\fancyplain{}{\bfseries\thepage}}\cfoot{}

%%%%%%%%%%%%%%%%%%%%%%%%%%%%%%%%%%%%%%%%%%%%%%%%%%%%%%%%%%%%%%%%%%%%%%%%%%%%%%%%
\selectlanguage{portuges}

\Sumario
\ListadeSimbolos
\listoffigures
\listoftables
\lstlistoflistings %lista de codigos fonte - Para inserir a listagem de codigos fonte
\newpage
\cleardoublepage

\Introducao


%%%%%%%%%%%%%%%%%%%%%%%%%%%%%%%%%%%%%%%%%%%%%%%%%%%%%%%%%%%%%%%%%%%%%%%%%%%%%%%%
%
% Hifenizacao - Colocar lista de palavras que nao devem ser separadas e que 
% nao estao no dicionario portugues.
% As palavras do dicionario portugues ja sao separadas corretamente pelo lateX
%
\hyphenation{hardware software}


%%%%%%%%%%%%%%%%%%%%%%%%%%%%%%%%%%%%%%%%%%%%%%%%%%%%%%%%%%%%%%%%%%%%%%%%%%%%%%%%
%% A partir daqui coloque seus capitulos. Sugere-se que eles sejam inseridos com o comando \input
%% Da seguinte maneira:
%%
%% \input{cap1} 
%% \input{cap2}
\selectlanguage{brazilian}

\input{cap1} 
% (5-10) páginas
%
%   Objetivos geral e específicos
%   Resultados esperados
%   Limitações do trabalho
%   Métodos
%   Justificativa
%   Descrição dos demais capítulos

\chapter{Introdução}

\section{Estudos Empíricos na Engenharia de Software}

A importância de estudos empíricos na ciência da computação tem sido reconhecida por diversos pesquisadores nos últimos anos \cite{Basili1996,Tichy1998,Feitelson2005}. No caso específico da engenharia de software, estudos empíricos frequentemente precisam mobilizar equipes de programadores e outros especialistas, o que torna os experimentos caros, difíceis de reproduzir e pouco generalizáveis.

Uma área da engenharia de software que demanda estudos empíricos é a \emph{engenharia reversa}. Engenharia reversa é qualquer atividade cujo objetivo é recuperar conhecimento sobre um sistema de software existente para apoiar a execução de uma tarefa de engenharia de software, por exemplo, a modificação de um sistema \cite{Tonella2007}. Atividades de engenharia reversa incluem localizar partes de um programa que implementam determinada funcionalidade \cite{Zhao2004}, prever os impactos de uma mudança \cite{Arnold1993} e recuperar aspectos não documentados da arquitetura de um sistema \cite{Pollet2007}.

Ferramentas de engenharia reversa em geral fornecem resultados imprecisos. O grau de acerto pode ser medido através de experimentos em que as ferramentas são aplicadas a diversos sistemas de software e, a seguir, especialistas nos sistemas validam os resultados encontrados. Muitas vezes é inviável, no entanto, aplicar esse tipo de experimento a uma amostra significativa de sistemas, dada a necessidade de se mobilizar um especialista para cada sistema.

Nas situações em que experimentos são inviáveis, cientistas recorrem a modelos. Modelos são formalizações simplificadas da realidade e, como tais, são usados para prever resultados que seriam obtidos com dados reais. Por um lado, modelos embutem suposições possivelmente incorretas sobre a realidade; por outro lado, a simulação através de modelos pré-existentes é um processo barato, controlado e repetível.

Nesta dissertação, o problema da carência de estudos experimentais sobre técnicas de \emph{agrupamento de software} --- uma forma de engenharia reversa --- é abordado através da simulação de modelos de sistemas de software. A seguir o problema é descrito em detalhes para então ser enunciado o objetivo deste trabalho.

\section{Agrupamento de Software}

% A divisão conceitual de um sistema de software em módulos é uma informação valiosa durante o seu desenvolvimento. Uma boa organização modular revela subconjuntos de um sistema que podem ser desenvolvidos por equipes trabalhando de forma mais ou menos independente, o que contribui para reduzir o tempo de implementação. Apesar disso, o conhecimento sobre a organização de um sistema muitas vezes é mal documentado e acaba se perdendo à medida que os desenvolvedores são substituídos \cite{Clements2002}.

O tempo gasto na manutenção de um sistema de software pode ser afetado pela atribuição de desenvolvedores a entidades da implementação de um sistema (arquivos fonte, classes, funções etc.). Se duas equipes de desenvolvimento são responsáveis por manter entidades altamente interdependentes (com alto acoplamento), elas precisam coordenar seu trabalho frequentemente para evitar duplicação de código e introdução de defeitos.

Portanto, para otimizar o tempo de desenvolvimento de um sistema, um gerente de desenvolvimento pode agrupar entidades que compõem o sistema em módulos, ou grupos de entidades, de forma que o acoplamento entre módulos seja minimizado. Então, cada módulo é atribuído a uma equipe e, por causa do baixo acoplamento entre módulos, a necessidade de comunicação entre equipes é reduzida, o que acelera o desenvolvimento.

Para organizar um sistema em módulos, um gerente precisa ter conhecimento global das dependências entre as entidades que formam o sistema. Embora ferramentas de análise de dependências possam extrair tal informação a partir do código-fonte do sistema, a tarefa de encontrar um bom mapeamento entre entidades e módulos pode ser excessivamente difícil devido ao grande número de entidades e dependências que precisam ser consideradas.

Algoritmos de agrupamento de software podem facilitar a tarefa, ao agrupar entidades em módulos de maneira a reduzir o número de dependências entre módulos. Uma forma de avaliar algoritmos de agrupamento de software é aplicá-los a uma coleção de sistemas de software com agrupamentos de referência feitos por desenvolvedores experientes \cite{Anquetil1999}. Sob esse critério, bons algoritmos são aqueles que acham agrupamentos similares aos agrupamentos de referência.

Infelizmente, existem poucos agrupamentos de referência disponíveis publicamente \cite{Koschke2000} e, consequentemente, poucos estudos empíricos sobre algoritmos de agrupamento de software. Além disso, como o custo de se obter agrupamentos de referência para grandes sistemas é alto, muitos estudos testam os algoritmos com sistemas pequenos e médios \cite{Anquetil1999,Maqbool2007,Bittencourt2009}.

\section{Modelagem e Simulação para Avaliar Algoritmos de Agrupamento}

Propomos, neste trabalho, avaliar algoritmos de agrupamento de software com sistemas de software sintéticos, gerados a partir da simulação de modelos de software. Para tanto, os modelos devem gerar sistemas de software tendo como base uma descrição dos módulos do sistema. O sistema, portanto, já é gerado com um agrupamento de referência.

De posse desse modelo, a avaliação de algoritmos de agrupamento de software consiste nos seguintes passos: (i) gerar, através de um modelo, diversos sistemas de software sintéticos a partir de módulos previamente estabelecidos; (ii) aplicar algoritmos de agrupamento sobre os sistemas sintéticos; (iii) comparar os agrupamentos encontrados pelos algoritmos com os agrupamentos de referência.

Os sistemas gerados pelos modelos não precisam ser detalhados no nível de implementação, uma vez que muitos algoritmos de agrupamento de software operam sobre grafos que representam as dependências entre entidades de implementação. Desta forma, é suficiente considerar modelos que produzem grafos e tratar os grafos como uma representação das dependências encontradas na implementação de um sistema.

\section{Objetivo}

Uma possível objeção para o uso de modelos é o argumento de que os grafos gerados pelos modelos seriam muito diferentes de grafos extraídos de sistemas de software reais e, portanto, conclusões obtidas com estes primeiros não poderiam ser generalizadas para estes últimos.

Com o propósito de suprir a carência de sistemas de software com agrupamentos de referência para testar algoritmos de agrupamento, o objetivo deste trabalho é encontrar modelos que geram grafos organizados em módulos similares a grafos de dependências extraídos de sistemas de software reais. O principal desafio nesta tarefa é mostrar que os grafos sintéticos se assemelham a grafos extraídos de sistemas de software reais, quando for o caso.

% \subsection{Objetivos Específicos}
% 
% Para alcançar o objetivo desta pesquisa, é necessário alcançar os seguintes objetivos específicos:
% 
% \begin{itemize}
% 	\item encontrar modelos que geram grafos organizados em módulos;
% 	\item avaliar os modelos quanto à similaridade entre os grafos gerados e grafos extraídos de sistemas de software;
% \end{itemize}

\section{Estrutura da Dissertação}

A estrutura do restante desta dissertação é descrita a seguir. 
%
No Capítulo \ref{cap:agrupamento} são descritos alguns algoritmos de agrupamento de software, bem como uma abordagem de avaliação dos algoritmos.
%
No Capítulo \ref{cap:redes} são apresentados estudos recentes sobre a estrutura de grafos extraídos de sistemas de software e grafos estudados em outros domínios, bem como modelos que procuram explicar a formação de grafos organizados em módulos.
%
No Capítulo \ref{cap:avaliacao} é relatado um experimento que mostra que os modelos apresentados geram grafos similares a grafos extraídos de software.
%
No Capítulo \ref{cap:estudo} é relatado um experimento no qual algoritmos de agrupamento são aplicados a grafos sintéticos.
%
No Capítulo \ref{cap:conclusao} são apresentadas as conclusões deste trabalho. 

%%%%%%%%%%%%%%%%%%%%%%%%%
% definir Engenharia Reversa.

% % O PROBLEMA FUNDAMENTAL: ATRIBUIÇÃO DE TRABALHO (fold)
% 
% Sistemas de software grandes precisam ser divididos em pedaços para que possam ser mantidos por equipes de programadores.
% 
% De nada adianta ter muitos desenvolvedores se a separação entre os módulos nos quais eles trabalham não é clara. Fred Brooks: adding more developers to a late project renders it later. Custo de comunicação. % ver versões antigas do artigo do csmr2010
% 
% Em alguns projetos, a estrutura do sistema e a estrutura de desenvolvedores é bem documentada (arquitetura de software).
% 
% Em outros casos, o software cresce de forma que o conhecimento da estrutura global se perde. Ou, ainda, pode se estar fazendo a manutenção de um sistema de software legado, no qual os desenvolvedores originais não estão disponíveis.
% 
% A aquisição do conhecimento sobre a estrutura global do software é custosa.
% 
% Ferramentas de clustering ajudam a achar uma boa separação de módulos, facilitando a atribuição de trabalho.
% 
% Recuperação de arquitetura de software é uma área bastante ampla, como será mostrado na Seção XX. Um significado específico será apresentada na seção YY.
% % explicar por que evito o termo arquitetura e me concentro em clustering de software, conceito que deve ser definido em um capítulo posterior.
% 
% %%%%%%%%%%%%%%%%%%%%%%%%% (end)
% 
% % O PROBLEMA DERIVADO: FALTA DE VALIDAÇÃO EMPÍRICA NA ENG DE SOFTWARE (fold)
% % (e em particular na área de recuperação de arquitetura)
% 
% Especialistas têm sentido falta de validação empírica na área de engenharia de software. Feitelson, Basili, Tichy. Não é à toa. Experimentos controlados em ES são caros, difíceis de reproduzir e pouco gerais. Mobilizam equipes de programadores. % ver proposta de mestrado
% 
% Esse cenário se repete na área de recuperação de arquitetura. No máximo há estudos de caso.
% 
% Nesses estudos de caso, são estudados sistemas que possuem uma estrutura modular de referência ou então são mobilizados especialistas para chegar a uma modularização (Koschke). NÃO HÁ BENCHMARKS! % ver WDCOPIN
% 
% Em outras áreas da ciência da computação, como redes sistemas distribuídos, é comum recorrer a modelos e simulações. Modelos são simplificações, mas a simulação é uma abordagem barata (tendo um modelo), controlada e repetível. % ver proposta de mestrado
% 
% %%%%%%%%%%%%%%%%%%%%%%%%% (end)
% 
% % O OBJETIVO DESTE TRABALHO: ESTUDAR REC. DE ARQ. COM MODEL. E SIMULAÇÃO! (fold)
% 
% Descobrir modelos de software que apoiem a avaliação de algoritmos de clustering.
% 
% Objetivos específicos:
%  ...
% 
% %%%%%%%%%%%%%%%%%%%%%%%%% (end)
% 
% % LIMITAÇÕES % (fold)
% 
% Algoritmos de clustering aplicados a programas OO representados como um grafo de dependências estáticas.
% 
% %%%%%%%%%%%%%%%%%%%%%%%%% (end)
% 
% % METODOLOGIA (fold)
% Métodos estatísticos, física estatística.
% 
% %%%%%%%%%%%%%%%%%%%%%%%% (end)


% (15-25)
% 
%     * redes complexas
%     * designs de software como redes complexas
%     * conceito de tríadas como base para caracterização de redes complexas
%     * clustering
%     * avaliação de algoritmos de clustering

\chapter{Fundamentação Teórica} 


% 10-30 páginas

\chapter{Modelos de Síntese de Designs}
% Avaliação da ("softwareness") Redes de Software e outros tipos de redes
% complexas (5-10)
%	
%		 * as redes geradas são verossímeis?

\chapter{Avaliação dos Modelos}
% (5-20)

\chapter{Estudo de Caso: Comparando Algoritmos de Clustering}

Caracterização do experimento segundo framework de Basili.

Objetivos (perguntas a responder):
1. ...
2. ...

Método:
Design fatorial

..
..
..

Resultados:

..
 
% 2-10 páginas

%   (Limitações)
%   (Lições aprendidas)
% Recapitulação
% Resumo das contribuições
% Trabalhos futuros

\chapter{Conclusão}

Contribuições
  ...
  Publicação no CSMR 2010!
  Modelo BCR+

Limitações

Trabalhos futuros
  Refinar modelos, com informações hierárquicas, tipos de vértices, múltiplas arestas com tipos e peso...
  Refinar modelos, adicionando a dimensão da evolução
  Pesquisar outros problemas, além de clustering de software, que possam se beneficiar de modelos baseados nos modelos aqui apresentados.

%%%%%%%%%%%%%%%%%%%%%%%%%%%%%%%%%%%%%%%%%%%%%%%%%%%%%%%%%%%%%%%%%%%%%%%%%%%%%%
%% BIbliografia
%% Coloque suas referencias no arquivo ref.bib e descomente as proximas duas linhas

\bibliographystyle{plain} % estilo de bibliografia   plain,unsrt,alpha,abbrv.
\bibliography{dissertacao} % arquivos com as entradas bib.

%%%%%%%%%%%%%%%%%%%%%%%%%%%%%%%%%%%%%%%%%%%%%%%%%%%%%%%%%%%%%%%%%%%%%%%%%%%%%%%%
%% Apendice
% Caso seja necessario algum apendice, descomente a proxima linha.

\appendix
\begin{chapter}{Redes Empregadas na Avaliação dos Modelos} \label{cap:lista-redes}
% select nme_network, to_char(s_score, '0.99'), n_vertices, n_edges 
% from view_network
% where fk_classification = 1
% OR nme_network ilike '%irpf%'
%

\begin{center}
\begin{longtable}{| p{10cm} | c | r | r |}
	\caption{65 redes de software (conjunto $T$).}	\\
	\hline
	\textbf{Sistema} & \textbf{S($x$, $T$)} & \textbf{Vértices} & \textbf{Arestas} \\ \hline
	\hline
	AbaGuiBuilder-1.8 &  $0,97$ & 2929 & 12399 \\ \hline
	alfresco-labs-deployment-3Stable &  $0,97$ & 2441 & 9178 \\ \hline
	aoi272 &  $0,98$ & 1579 & 11670 \\ \hline
	ArgoUML-0.28 &  $0,97$ & 5075 & 29018 \\ \hline
	battlefieldjava-0.1 &  $0,98$ & 707 & 2038 \\ \hline
	broker-4.1.5 &  $0,97$ & 5940 & 25589 \\ \hline
	checkstyle-5.0 &  $0,98$ & 1174 & 4513 \\ \hline
	dbwrench &  $0,98$ & 2015 & 10188 \\ \hline
	dom4j-1.6.1 &  $0,98$ & 4666 & 31689 \\ \hline
	ec2-api-tools &  $0,98$ & 5255 & 26916 \\ \hline
	ermodeller-1.9.2-binary &  $0,97$ & 1441 & 6261 \\ \hline
	findbugs-1.3.8 &  $0,98$ & 3401 & 20023 \\ \hline
	flyingsaucer-R8 &  $0,97$ & 1323 & 6610 \\ \hline
	freetts-1.2.2-bin &  $0,91$ & 273 & 1106 \\ \hline
	ganttproject-2.0.9 &  $0,98$ & 5447 & 25558 \\ \hline
	gdata-src.java-1.31.1 &  $0,97$ & 1649 & 8150 \\ \hline
	GEF-0.13-bin &  $0,97$ & 352 & 1484 \\ \hline
	geoserver-2.0-beta1-bin &  $0,97$ & 25007 & 145034 \\ \hline
	geotools-2.5.5-bin &  $0,97$ & 13559 & 79499 \\ \hline
	gfp\_0.8.1 &  $0,87$ & 475 & 1030 \\ \hline
	guice-2.0 &  $0,98$ & 638 & 2980 \\ \hline
	gwt-windows-1.6.4 &  $0,98$ & 6065 & 32729 \\ \hline
	hibernate-distribution-3.3.1.GA-dist &  $0,98$ & 4340 & 22410 \\ \hline
	Hl7Comm.1.0.1 &  $0,97$ & 337 & 1335 \\ \hline
	hsqldb\_1\_8\_0\_10 &  $0,95$ & 369 & 1901 \\ \hline
	iBATIS\_DBL-2.1.5.582 &  $0,95$ & 291 & 1300 \\ \hline
	iFreeBudget-2.0.9 &  $0,98$ & 4053 & 20028 \\ \hline
	iReport-nb-3.5.1 &  $0,97$ & 35019 & 184701 \\ \hline
	IRPF2009v1.1 &  $0,98$ & 5396 & 29247 \\ \hline
	JabRef-2.5b2-src &  $0,98$ & 2473 & 10437 \\ \hline
	jai-1\_1\_4-pre-dr-b03-lib-linux-i586 &  $0,96$ & 936 & 5086 \\ \hline
	jailer\_2.9.9 &  $0,97$ & 3691 & 14890 \\ \hline
	jakarta-tomcat-5.0.28-embed &  $0,98$ & 2314 & 10652 \\ \hline
	jalopy-1.5rc3 &  $0,98$ & 888 & 4349 \\ \hline
	jasperreports-3.5.2-project &  $0,98$ & 20508 & 120243 \\ \hline
	jfreechart-1.0.13 &  $0,97$ & 1989 & 13891 \\ \hline
	jGnash-2.2.0 &  $0,98$ & 6948 & 38303 \\ \hline
	jgraphpad-5.10.0.2 &  $0,96$ & 519 & 2257 \\ \hline
	jmsn-0.9.9b2 &  $0,83$ & 232 & 777 \\ \hline
	juel-2.1.2 &  $0,94$ & 111 & 435 \\ \hline
	juxy-0.8 &  $0,97$ & 3025 & 20427 \\ \hline
	JXv3.2rc2deploy &  $0,96$ & 711 & 2573 \\ \hline
	makagiga-3.4 &  $0,98$ & 1999 & 10819 \\ \hline
	MegaMek-v0.34.3 &  $0,98$ & 2855 & 19575 \\ \hline
	mondrian-3.1.1.12687 &  $0,97$ & 2000 & 14128 \\ \hline
	myjgui\_0.6.6 &  $0,96$ & 604 & 2430 \\ \hline
	oddjob-0.26.0 &  $0,98$ & 4204 & 18752 \\ \hline
	openxava-3.1.2 &  $0,98$ & 16919 & 89049 \\ \hline
	pdfsam-1.1.3-out &  $0,97$ & 3043 & 16024 \\ \hline
	peer-4.1.5 &  $0,97$ & 9291 & 50614 \\ \hline
	pentaho-reporting-engine-classic-0.8.9.11 &  $0,98$ & 3454 & 17948 \\ \hline
	pjirc\_2\_2\_1\_bin &  $0,84$ & 180 & 687 \\ \hline
	pmd-bin-4.2.5 &  $0,98$ & 1719 & 8098 \\ \hline
	proguard4.3 &  $0,98$ & 604 & 4923 \\ \hline
	rapidminer-4.4-community &  $0,96$ & 11764 & 55976 \\ \hline
	smc\_6\_0\_0 &  $0,97$ & 698 & 3571 \\ \hline
	squirrel-sql-3.0.1-base &  $0,98$ & 8982 & 43806 \\ \hline
	squirrel-sql-3.0.1-standard &  $0,98$ & 11922 & 53969 \\ \hline
	stendhal-0.74 &  $0,97$ & 747 & 3056 \\ \hline
	subethasmtp-3.1 &  $0,98$ & 1485 & 8174 \\ \hline
	thinkui\_sqlclient-1.1.2 &  $0,97$ & 2883 & 21979 \\ \hline
	tvbrowser-2.7.3-bin &  $0,97$ & 3335 & 14110 \\ \hline
	villonanny-2.3.0.b02.bin &  $0,98$ & 1949 & 6708 \\ \hline
	worker-4.1.5 &  $0,97$ & 5940 & 25589 \\ \hline
	zk-bin-3.6.1 &  $0,95$ & 13911 & 82219 \\ \hline
	
	% \caption{\label{tab:redes}Sistemas de software usados no estudo de software-realismo de modelos e respectivos valores S.}
\end{longtable}
\end{center}


\begin{center}
\begin{longtable}{| p{6cm} | l | r | r | }
	\caption{66 redes de domínios diversos (conjunto $\bar{T}$).} \\
	\hline
	\textbf{Rede} & \textbf{S($x$, $T$)} & \textbf{Vértices} & \textbf{Arestas} \\ \hline
	\hline
	5 redes de amizade do site Facebook \cite{Traud2008} & $-0,130 \pm 0,007$ & 768 a 18162 & 33312 a 1533600 \\ \hline
	3 redes de circuitos eletrônicos \cite{Milo2004} & $0,433 \pm 0,005$ & 121 a 511 & 189 a 819 \\ \hline
	4 redes de adjacências entre palavras \cite{Milo2004} & $0,573 \pm 0,064$ & 2703 a 11585 & 8300 a 46281 \\ \hline
	3 redes de estrutura proteica \cite{Milo2004} & $0,540 \pm 0,046$ & 52 a 96 & 123 a 213 \\ \hline
	2 redes sociais de sentimento positivo \cite{Milo2004} & $0,555 \pm 0,248$ & 31 a 66 & 96 a 182 \\ \hline
	43 redes metabólicas \cite{Jeong2000} & $0,509 \pm 0,015$ & 408 a 2360 & 792 a 5959 \\ \hline
	Rede de interação entre proteínas da levedura \cite{Jeong2001} & $0,22$ & 687 & 1079 \\ \hline
	Links entre blogs políticos \cite{Adamic2005} & $0,96$ & 1489 & 19090 \\ \hline
	Rede neural do verme C Elegans \cite{Watts1998} & $0,87$ & 296 & 2359 \\ \hline
	Rede ``beta3sreduced'' (fonte desconhecida) & $0,05$ & 1286 & 33813 \\ \hline
	Rede ``czech'' (fonte desconhecida) & $0,78$ & 5225 & 51687 \\  \hline
	Rede ``ecoli-metabolic'' (fonte desconhecida) & $0,69$ & 895 & 964 \\ \hline
	
	% \caption{\label{tab:redes-outros}Redes de domínios diversos usados no estudo de software-realismo de modelos e respectivos valores S.}
\end{longtable}
\end{center}

\end{chapter}

\begin{chapter}{Algoritmos de Agrupamento} \label{cap:agrupamento}

A seguir são descritos brevemente três tipos de algoritmos de agrupamento de software que operam sobre redes de dependências entre entidades. % Para esta pesquisa foram escolhidos algoritmos estudados por diferentes grupos de pesquisa e com implementações disponíveis publicamente.

\begin{section}{Algoritmos Hierárquicos Aglomerativos}

Algoritmos hierárquicos aglomerativos têm sido aplicados a diversos problemas, incluindo o agrupamento de software \cite{Anquetil1999,Maqbool2007}. Essa família de algoritmos funciona com qualquer descrição de entidade, desde que seja fornecida uma métrica de similaridade entre pares de entidades. A similaridade entre duas entidades é medida em uma escala contínua de 0 a 1. O algoritmo é simples: inicia-se com um módulo para cada entidade e então mesclam-se sucessivamente os dois módulos mais similares até restar apenas um módulo com todas as entidades.

No contexto de recuperação de arquitetura de software, é comum descrever as entidades de código-fonte como um grafo de dependências entre entidades e usar a métrica de Jaccard para medir a similaridade entre duas entidades \cite{Anquetil1999}. Sejam X e Y duas entidades, e seja d(A) o conjunto de entidades ligadas à entidade A no grafo de dependências. A similaridade entre duas entidades, X e Y, é dada pela expressão a seguir:

$$
\mathrm{sim}(X, Y) ~=~ \frac{|\mathrm{d}(X) \cap \mathrm{d}(Y)|}{|\mathrm{d}(X) \cup \mathrm{d}(Y)|}
$$

O que diferencia os algoritmos hierárquicos aglomerativos entre si é o critério usado para medir a similaridade entre dois módulos. Os dois critérios mais estudados no contexto de agrupamento são chamados de ligação simples (SL, do inglês \emph{single linkage}) e ligação completa (CL, do inglês \emph{complete linkage}). Na ligação simples, a similaridade entre dois módulos é computada como a maior similaridade entre pares de entidades tiradas uma de cada módulo. Na ligação completa é considerada a menor similaridade.

Naturalmente, um agrupamento que consiste de apenas um módulo com todas as entidades analisadas é de pouco valor. Por essa razão deve existir um critério de parada que interrompa o algoritmo antes de todos os módulos terem sido mesclados em um grande módulo. No contexto de agrupamento de software, o critério mais comumente usado é a altura de corte, $h$, que varia entre 0 e 1. Sob esse critério, o algoritmo interrompe sua execução quando a maior similaridade entre dois módulos é menor ou igual a $1 - h$. Assim, quanto maior a altura de corte $h$, menor o número de módulos do agrupamento resultante.
	
\end{section}

\begin{section}{Bunch}

Com a finalidade de auxiliar a compreensão de programas, o algoritmo Bunch \cite{Mancoridis1998} agrupa o conjunto de entidades de um programa de acordo com os relacionamentos existentes entre elas, representados como um grafo orientado. O agrupamento é tratado como um problema de otimização no qual o objetivo é maximizar uma função denominada qualidade de modularização (QM), que recompensa agrupamentos com muitas arestas internas (que ligam entidades de um mesmo módulo) e poucas arestas externas (que ligam entidades de módulos distintos).

% TODO: fórmula da QM

Encontrar um agrupamento que possui QM ótima é um problema intratável; por isso o algoritmo Bunch usa heurísticas como algoritmos genéticos e \emph{hill-climbing} para obter resultados quase ótimos.

\end{section}

\begin{section}{ACDC}

O algoritmo ACDC \cite{Tzerpos2000} foi projetado com o intuito de encontrar agrupamentos que facilitam a compreensão de programas. Assim como o Bunch, ele opera sobre grafos orientados. Sua principal característica é a identificação de conjuntos dominantes de vértices no grafo. Um conjunto dominante é um conjunto de vértices, $v_0, v_1, \ldots{}, v_n$, onde $v_0$ é o vértice dominante, que satisfaz a duas condições: (i) existe um caminho entre $v_0$ e qualquer $v_i$; (ii) qualquer caminho de um vértice que não pertence ao conjunto até um vértice do conjunto, $v_i$, passa por $v_0$. O algoritmo ACDC identifica os conjuntos dominantes, em ordem crescente de número de vértices, e considera-os módulos do agrupamento. 

% subgraph dominator
% evita módulos muito grandes.
	
\end{section}	

\end{chapter}


% \begin{chapter}{Métrica MoJoSim} \label{cap:mojosim}
% 
% A métrica MoJoSim \cite{Bittencourt2009}, baseada na métrica MoJo \cite{Tzerpos1999}, mede a similaridade entre dois agrupamentos, um dos quais é considerado agrupamento de referência. Neste capítulo a métrica MoJo é explicada através de um exemplo e, a seguir, a métrica MoJoSim é definida.
% 
% % TODO: reescrever sem emoção, afinal isto virou um apêndice. Talvez mover o início para o capítulo do estudo.
% 
% % Como já foi mencionado, não existe um critério objetivo para determinar qual é o melhor agrupamento das entidades de um sistema de software. Ainda assim, alguma forma de avaliação de algoritmos de agrupamento é necessária para comparar os diferentes algoritmos. Uma das formas de avaliar os agrupamentos produzidos por algoritmos comparando-os com agrupamentos de referência, produzidos por especialistas, do conjunto de sistemas nos quais os algoritmos foram executados \cite{Koschke2000}.
% 
% A Figura \ref{fig:mojo} ilustra dois agrupamentos de um mesmo sistema de software hipotético, representado como um grafo orientado que descreve dependências estáticas entre classes do sistema. É fácil perceber que os agrupamentos não são idênticos: um deles possui quatro módulos, enquanto o outro possui três. Ainda assim, os agrupamentos são bastante semelhantes. O módulo $M_4$ corresponde exatamente ao módulo $M_C$; O módulo $M_3$ é quase igual ao módulo $M_B$; e os módulos $M_1$ e $M_2$, unidos, são parecidos com o módulo $M_A$.
% 
% \begin{figure}[htbp]
% 	\centering
% 		\includegraphics[scale=1]{figuras/redes-dupla}
% 	\caption{Dois agrupamentos de um mesmo sistema de software hipotético, descrito como um grafo que representa dependências entre classes.}
% 	\label{fig:mojo}
% \end{figure}
% 
% % É fácil verificar se dois agrupamentos de um mesmo conjunto de entidades são idênticos ou não. Para comparar diferentes agrupamentos produzidos por algoritmos, no entanto, é preciso definir uma métrica de similaridade entre agrupamentos que seja capaz de determinar qual agrupamento de uma coleção de agrupamentos é mais similar a um agrupamento de referência.
% 
% A ideia por trás da métrica MoJo é que um agrupamento dado é muito similar a um agrupamento de referência se é possível transformar o primeiro agrupamento no segundo usando poucas operações do tipo mover e mesclar. A operação mover consiste de mover uma entidade de um módulo para outro módulo distinto. A operação mesclar consiste de mesclar dois módulos. O MoJo entre  dois agrupamentos é igual ao número de operações de mover e mesclar que são necessárias para transformar o primeiro agrupamento no segundo. Quanto menor o MoJo, maior a similaridade entre dois agrupamentos; agrupamentos idênticos têm MoJo igual a zero.
% 
% % A métrica MoJo não é simétrica. Para chegar a essa conclusão, basta observar que há uma operação de mesclar módulos, mas não uma operação de dividir um módulo em dois. No contexto de avaliação de algoritmos de agrupamento de software, é considerado o número de operações necessárias para transformar o agrupamento encontrado por um algoritmo no agrupamento de referência, e não o contrário.
% 
% Na Figura \ref{fig:mojo}, considerando que o agrupamento de referência é o segundo (o da direita), o MoJo entre os dois agrupamentos vale 2. Para transformar o primeiro agrupamento no segundo são necessárias, portanto, duas operações: mesclar os módulos $M_1$ e $M_2$, e mover o vértice $v_5$ para o módulo $M_3$. Após a realização dessas operações, os agrupamentos se tornam idênticos, considerando as correspondências $(M_1 \cup M_2) = M_A$, $M_3 = M_B$ e $M_4 = M_C$.
% 
% Observando que o valor MoJo entre dois agrupamentos com o mesmo número de entidades varia entre 0 e o número de entidades em cada agrupamento, $n$, é possível definir uma métrica, chamada MoJoSim, que mapeia o valor MoJo em uma escala de 0 a 1 \cite{Bittencourt2009}. O valor 0 representa a menor similaridade e o valor 1, a maior similaridade. O valor de MoJoSim entre dois agrupamentos é definido de acordo com a seguinte equação:
% 
% $$
% \mathrm{MoJoSim}(X, Y) ~=~ 1 - \frac{\mathrm{MoJo}(X, Y)}{n}
% $$
% 
% O valor de MoJoSim na Figura \ref{fig:mojo} é, portanto, igual a $1 - \frac{2}{10} = 0,8$.
% 
% % Anquetil avaliou 
% 
% % Movido
% % Wu, Hassan e Holt \cite{Wu2005} avaliaram os algoritmos Bunch, ACDC e algoritmos aglomerativos comparando, através da métrica MoJo, os agrupamentos encontrados pelos algoritmos com agrupamentos de referência de 5 sistemas de software em C e C++, representados como um grafo dos arquivos fonte e dependências estáticas entre os arquivos. Os agrupamentos de referência foram obtidos automaticamente a partir de uma análise da estrutura de diretórios dos sistemas. Cada diretório contendo pelo menos 5 arquivos fonte foi considerado um módulo do sistema. Bittencourt e Guerrero \cite{Bittencourt2009} realizaram um experimento semelhante, porém avaliando algoritmos de agrupamento estudados em outros domínios aplicados sobre um conjunto de 4 sistemas em Java.
% 
% \end{chapter}

%%%%%%%%%%%%%%%%%%%%%%%%%%%%%%%%%%%%%%%%%%%%%%%%%%%%%%%%%%%%%%%%%%%%%%%%%%%%%%%%

\end{document}

% GLOSSÁRIO:
% * Algoritmos de AGRUPAMENTO de software (programas de computador)
% * Lei de potência
% * ENTIDADES de código-fonte
% * REDE ORGANIZADA EM MÓDULOS.
% * GERAR. REDES SINTÉTICAS.
% * REDE DE SOFTWARE.
% * SOFTWARE-REALISMO.
% * Desempenho (no sentido de qualidade do agrupamento)


% * Grafo particionado? Rede com módulos embutidos? Rede estruturada em módulos? Rede hierárquica de dois níveis? Rede dividida em módulos.

% Introdução
% Objetivo
% Material e Métodos
% Resultados
% Conclusão
% 
% %%%%%%%%%%%%%%%%%%%%%%%%%%%%%%%%%%%%%%%%
% 
%     *  Introdução. O problema do desenvolvimento de grandes sistemas; técnicas de recuperação de arquitetura como parte da solução. Dificuldade de se avaliar essas técnicas. Proposta do trabalho: avaliação através de redes de software sintéticas. Objetivo: avaliar se redes sintéticas se assemelham às reais.
%     * Conceitos. Recuperação de arquitetura. Redes de software. Redes complexas. Modelos de redes.
%     * Trabalhos relacionados. Revisão crítica de trabalhos de avaliação empírica de técnicas de recuperação arquitetural.
%     * Avaliação empírica de modelos de redes. Experimento que mede a similaridade entre redes sintetizadas pelos modelos e redes reais. Discussão dos métodos e resultados.
%     * Estudo de caso. Aplicação da avaliação proposta para estudar técnicas pré-selecionadas de recuperação de arquitetura. Comparação dos resultados obtidos em trabalhos relacionados.
%     * Conclusão. Recapitulação, contribuições e possíveis trabalhos futuros.
%     * Referências. Referências bibliográficas.
% 
% %%%%%%%%%%%%%%%%%%%%%%%%%%%%%%%%%%%%%%%%
% 
% Resumo (inclui resultados)
% Introdução
%   Objetivos geral e específicos
%   Resultados esperados
%   Limitações do trabalho
%   Métodos
%   Justificativa
%   Descrição dos demais capítulos
% Revisão bibliográfica (Fundamentação teórica)
%   Conceitos
%   Trabalhos correlatos
%   (Evolução das ideias)
% Desenvolvimento
%   Definições (constitutivas/operacionais)
%   Métodos?
%   Apresentação e análise dos dados
% Conclusões (liga desenvolvimento a introdução)
%   (Limitações)
%   (Lições aprendidas)
%   Recapitulação
%   Resumo das contribuições
%   Trabalhos futuros
% Referências
% 
