% (5-10) páginas
%
%   Objetivos geral e específicos
%   Resultados esperados
%   Limitações do trabalho
%   Métodos
%   Justificativa
%   Descrição dos demais capítulos

\chapter{Introdução}

%%%%%%%%%%%%%%%%%%%%%%%%%

% O PROBLEMA FUNDAMENTAL: ATRIBUIÇÃO DE TRABALHO (fold)

Sistemas de software grandes precisam ser divididos em pedaços para que possam ser mantidos por equipes de programadores.

De nada adianta ter muitos desenvolvedores se a separação entre os módulos nos quais eles trabalham não é clara. Fred Brooks: adding more developers to a late project renders it later. Custo de comunicação. % ver versões antigas do artigo do csmr2010

Em alguns projetos, a estrutura do sistema e a estrutura de desenvolvedores é bem documentada (arquitetura de software).

Em outros casos, o software cresce de forma que o conhecimento da estrutura global se perde. Ou, ainda, pode se estar fazendo a manutenção de um sistema de software legado, no qual os desenvolvedores originais não estão disponíveis.

A aquisição do conhecimento sobre a estrutura global do software é custosa.

Ferramentas de clustering ajudam a achar uma boa separação de módulos, facilitando a atribuição de trabalho.

Recuperação de arquitetura de software é uma área bastante ampla, como será mostrado na Seção XX. Um significado específico será apresentada na seção YY.
% explicar por que evito o termo arquitetura e me concentro em clustering de software, conceito que deve ser definido em um capítulo posterior.

%%%%%%%%%%%%%%%%%%%%%%%%% (end)

% O PROBLEMA DERIVADO: FALTA DE VALIDAÇÃO EMPÍRICA NA ENG DE SOFTWARE (fold)
% (e em particular na área de recuperação de arquitetura)

Especialistas têm sentido falta de validação empírica na área de engenharia de software. Feitelson, Basili, Tichy. Não é à toa. Experimentos controlados em ES são caros, difíceis de reproduzir e pouco gerais. Mobilizam equipes de programadores. % ver proposta de mestrado

Esse cenário se repete na área de recuperação de arquitetura. No máximo há estudos de caso.

Nesses estudos de caso, são estudados sistemas que possuem uma estrutura modular de referência ou então são mobilizados especialistas para chegar a uma modularização (Koschke). NÃO HÁ BENCHMARKS! % ver WDCOPIN

Em outras áreas da ciência da computação, como redes sistemas distribuídos, é comum recorrer a modelos e simulações. Modelos são simplificações, mas a simulação é uma abordagem barata (tendo um modelo), controlada e repetível. % ver proposta de mestrado

%%%%%%%%%%%%%%%%%%%%%%%%% (end)

% O OBJETIVO DESTE TRABALHO: ESTUDAR REC. DE ARQ. COM MODEL. E SIMULAÇÃO! (fold)

Descobrir modelos de software que apoiem a avaliação de algoritmos de clustering.

Objetivos específicos:
 ...

%%%%%%%%%%%%%%%%%%%%%%%%% (end)

% LIMITAÇÕES % (fold)

Algoritmos de clustering aplicados a programas OO representados como um grafo de dependências estáticas.

%%%%%%%%%%%%%%%%%%%%%%%%% (end)

% METODOLOGIA (fold)
Métodos estatísticos, física estatística.

%%%%%%%%%%%%%%%%%%%%%%%% (end)
% ÍNDICE

O capítulo 1 trata disso, bla, bla
...
O capítulo N trata daquilo, bla, bla, bla, bla...
