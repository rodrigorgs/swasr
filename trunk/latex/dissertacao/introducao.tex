% (5-10) páginas
%
%   Objetivos geral e específicos
%   Resultados esperados
%   Limitações do trabalho
%   Métodos
%   Justificativa
%   Descrição dos demais capítulos

\chapter{Introdução}

% Estou puxando a brasa para engenharia reversa. Será que é necessário? Não é possível generalizar isso para engenharia de software?

As dependências entre entidades presentes no código-fonte de sistemas de software (classes, métodos, procedimentos, variáveis etc.) fornecem pistas sobre propriedades de sistemas de software, tais como facilidade de compreensão, facilidade de manutenção, facilidade de teste e facilidade de reuso. 
% Não surpreende que
A análise de tais dependências é essencial para o cálculo de diversas métricas de código-fonte, bem como para a realização de diversas atividades de engenharia reversa.

Engenharia reversa inclui ``qualquer método voltado para recuperar conhecimento sobre um sistema de software existente para apoiar a execução de uma tarefa de engenharia de software'' \cite{Tonella2007}. Nos métodos de engenharia reversa, as dependências podem ser usadas para auxiliar, por exemplo, a recuperação da arquitetura, a detecção de clones ou a localização de conceitos dentro de um sistema de software. 

Como em muitas outras disciplinas, a maturidade da disciplina de engenharia reversa requer que os métodos propostos sejam avaliados empiricamente. Um estudo realizado sobre artigos sobre engenharia reversa publicados de 2002 a 2005 revelou que 25\% dos artigos não apresentam qualquer forma de avaliação empírica e, dentre os demais artigos, 70\% apresentam apenas estudos de caso e relatos de experiência.

A escassez de experimentos controlados em engenharia reversa pode ser explicada, em parte, pelo alto custo envolvido na realização de experimentos de qualidade. Em muitos casos, os experimentos precisam envolver grupos de desenvolvedores balanceados segundo o grau de experiência e replicação. O custo de experimentação elevado, no entanto, não é exclusividade da engenharia reversa: ele ocorre também em áreas da computação como redes e sistemas distribuídos.

% Clarificar: modelos estatísticos? modelos computacionais? modelos estocásticos? modelos para simulação? modelagem científica?
Uma abordagem empregada quando os experimentos controlados são caros é a simulação de modelos. Na área de redes e sistemas distribuídos, é frequente o uso de ambientes de simulação de redes, que podem incorporar desde modelos de falhas de hardware até modelos de comportamento de usuários \cite{XXX}.
% An Integrated Experimental Environment for Distributed Systems and Networks. 
% comportamento: http://www.foibg.com/ijita/vol15/ijita15-1-p11.pdf

Na engenharia de software, no entanto, a abordagem de simulação é pouco usada. As aplicações se concentram na simulação de processos de software. Surpreendentemente, não há modelos para dependências entre entidades de sistemas de software.


The majority of research using simulation in software engineering is concerned with the simulation of software process. Prominent examples of this include the modelling of project planning [Kelln91], defect levels and staffing profiles [Raffo96] as well as system size and effort trends [Wern99]. These differ from the simulation model presented here in that they investigate the processes by which people, technology and practices are organized to transform information, materials and energy into a piece of software.



modelos de interação, modelos de distribuição de carga, 

métodos, técnicas e ferramentas


, tais como visualização, análise de impacto de mudanças, localização de conceitos e recuperação de arquitetura. 



A análise de dependências entre entidades ou elementos do código-fonte de um sistema de software é essencial para o cálculo de diversas métricas de código-fonte, bem como para a realização de atividades de engenharia reversa, tais como visualização, análise de impacto de mudanças, localização de conceitos e recuperação de arquitetura. 

a visualização da estrutura programas e inúmeras atividades de engenharia reversa, tais como de

% MOTIVAÇÃO / JUSTIFICATIVA

%

\begin{section}{Objetivos}
\end{section}

\begin{section}{Métodos}
	Teoria das redes complexas... métodos estatísticos...
\end{section}

\begin{section}{Resultados(?)}
\end{section}


\begin{section}{Estrutura desta Dissertação}	
\end{section}