% vim: tw=78 encoding=utf8
\documentclass{article}

\newcommand{\newrow}{\\\hline}
\newcommand{\x}{$\bullet$}

\usepackage[utf8]{inputenc}
\usepackage[brazil]{babel}

\newcommand{\review}[2]{\textcolor{red}{#1}\textcolor{blue}{#2}}

\title{
{\small Anteprojeto de Tese de Doutorado}
\\
Sobre a Adoção de Boas Práticas 
de Desenvolvimento de Software 
em Projetos de Software Livre
%Processos de Desenvolvimento em \\
%Software: Discurso e Prática
%
%Processos de Desenvolvimento em \\
%Projetos de Software Livre
}
\author{Rodrigo Rocha Gomes e Souza\\
\texttt{rodrigo@dcc.ufba.br}
}

\date{Dezembro de 2009}

\begin{document}

\sloppy
\maketitle
%\newpage
%\tableofcontents
%\newpage

%%%%%%%%%%%%%%%%%%%%%%%%%%%%%%%%%%%%%%%%%%%%%%%%%%%%%%%%%%%%%%%%%%%%%%%%%%%%%
%%%%%%%%%%%%%%%%%%%%%%%%%%%%%%%%%%%%%%%%%%%%%%%%%%%%%%%%%%%%%%%%%%%%%%%%%%%%%

% TODO: onion-like structure, hierarchical structure, core developers 

%   Processo afeta qualidade do software
%   Employee turnover is known to be high in the traditional software industry
% since many years ago. (rotatividade)
%   Turnover afeta processo?

% Apresentação de Terceiro
%   Problema
%   Objetivo
%   Usar abandono de desenvolvedores como critério de insucesso
%   Variáveis independentes, dependentes e de contexto

\section{Objetivo}

% TODO Trocar "prática" por "princípio" pra deixar "prática" apenas como
% contraponto de teoria?
%
% http://searchsoftwarequality.techtarget.com/sDefinition/0,,sid92_gci498678,00.html
% http://tlt-swg.blogspot.com/2006/08/best-practices-defined-usefully-tlt.html

No contexto de desenvolvimento de software, boas práticas são estratégias ou
atividades que, segundo mostra a experiência, contribuem para o sucesso de um
projeto de software. Apesar das vantagens, nem sempre é fácil aplicar boas
práticas em um projeto de software.
%A adoção de boas práticas em um projeto, naturalmente, depende de TUDO A
%decisão de adotar uma boa prática em um projeto

O principal objetivo desta pesquisa é entender como a adoção de boas práticas de
desenvolvimento de software é afetada por variações na forma de organização da
equipe de desenvolvedores e por características do projeto. Pretende-se assim
identificar em quais contextos determinadas práticas são efetivamente aplicadas
e entender os mecanismos que reforçam a adoção de práticas de desenvolvimento de
software.

A análise será focada nas atividades de implementação, documentação e controle
de qualidade. Serão investigadas, entre outras, as seguintes práticas:

\begin{itemize}
%  \item uso de testes de unidade;
  \item planejamento e implementação de testes antes da implementação das
  funcionalidades correspondentes;
  \item implementação de testes por pessoas diferentes daquelas que
  implementam funcionalidades;
  \item integração contínua com o repositório de código-fonte;
  \item correção de \emph{bugs} antes da implementação de novas funcionalidades;
  \item revisão de código.
\end{itemize}

% TODO: Mover rotatividade e variabilidade de experiência para motivação?
% TODO: Falar de centralidades (core developers etc.) antes de falar em
% rotatividade. Campo de influência de um desenvolvedor hub (concentrador)
Crowston e Howison \cite{crowston2005} recentemente mostraram que os padrões de
interação entre desenvolvedores variam bastante de projeto para projeto. Alguns
projetos são bastante centralizados em um conjunto pequeno de desenvolvedores
que interagem com boa parte dos demais; outros possuem uma organização mais
descentralizada, nas quais os desenvolvedores se organizam em grupos bem
definidos. Não é claro, no entanto, como práticas de desenvolvimento se
distribuem nessa rede social. O quanto variam as práticas entre diferentes
grupos de um projeto? 

% Há diferença de processos praticados por diferentes centros/subprojetos?
% Desenvolvedores centrais e nao centrais desempenham papeis tipicamente
% diferentes? (Capiluppi observou que desenvolvedores novos tendem a se
% concentrar em novas funcionalidades)

Sabe-se há muito tempo que é alta a rotatividade de desenvolvedores em empresas
de software, e pesquisas recentes mostram que o cenário também ocorre em
projetos de software livre \cite{robles2006}. Enquanto os processos praticados
por uma empresa são mantidos graças à organização hierárquica de seus
funcionários, vale a pena investigar como se dá essa manutenção em projetos de
software livre, especialmente os mais descentralizados, face à saída e à entrada
de desenvolvedores. Será que, na eventualidade da saída dos desenvolvedores mais
produtivos de um projeto, os demais desenvolvedores perpetuam as práticas
aplicadas por seus antecessores? Em que casos a saída de desenvolvedores pode
causar o abandono de práticas em um projeto?

Pretende-se também estudar diferenças entre as práticas adotadas por projetos
com características distintas. Características de projetos a serem estudadas
incluem grau de centralização, número de desenvolvedores e tamanho do software.
Esse estudo poderá levar a conclusões do tipo ``a prática \emph{X} não é
escalável: ela raramente é usada em projetos de grande porte''. 


%Em especial, as seguintes questões merecem FURTHER investigação:
%
%Ao investigar a resposta dessas questões em diversos projetos ao longo do
%tempo, se pretende chegar a respostas para as seguintes questões de pesquisa:
%
%\begin{itemize}
%  \item Em que condições a entrada e a saída de desenvolvedores de um projeto afeta o seu processo de desenvolvimento?
%  \item Existem diferenças nos processos aplicados por desenvolvedores com diversos níveis de envolvimento com o projeto?
%  \item Qual é o papel dos desenvolvedores mais envolvidos com um projeto na manutenção de seu processo de desenvolvimento?
%  \item Os subprojetos de um projeto compartilham um mesmo processo?
%***  \item Quais práticas são mais difundidas em projetos de software livre? Quais são menos difundidas?
%  \item Como se diferenciam os processos de projetos bem sucedidos e os processos de projetos que foram descontinuados?
%  \item Dentre os projetos mais bem sucedidos, de que forma os processos se diferenciam?
%  \item Como os processos se alteram nos diversos estágios do desenvolvimento de uma versão do software (alfa, beta, candidata a lançamento, em manutenção)?
%***  \item Existem práticas que são comumente aplicadas em conjunto?
%  \item Os projetos que dizem seguir um modelo de processo estão de fato aplicando as práticas associadas a ele?
%\end{itemize}

%%%%%%%%%%%%%%%%%%%%%%%%%%%%%%%%%%%%%%%%%%%%%%%%%%%%%%%%%%%%%%%%%%%%%%%%%%%%%
%%%%%%%%%%%%%%%%%%%%%%%%%%%%%%%%%%%%%%%%%%%%%%%%%%%%%%%%%%%%%%%%%%%%%%%%%%%%%

\section{Motivação}

% QUAL É O PROBLEMA?

% CONTEXTO O uso de boas práticas está associadas a melhores softwares. Em um
% ambiente com desenvolvedores com diversos níveis de experiência e alta
% rotatividade, a difusão de boas práticas pode uniformizar a qualidade do
% produto (software).
A aplicação de boas práticas de desenvolvimento está associada à melhoria na
qualidade do software produzido \cite{grady1993}. A difusão de boas práticas
dentro de um projeto é, ainda, uma forma de diminuir a diferença de
produtividade entre desenvolvedores menos experientes e desenvolvedores mais
experientes, o que contribui para tornar o processo de desenvolvimento de
software mais previsível.
%Há muito se discutem formas de tornar o processo de desenvolvimento de software
%mais previsível, aumentando a produtividade da equipe e a qualidade do
%software. Dessa discussão surgiram modelos de processos e cartilhas de boas
%práticas de desenvolvimento.

% PROBLEMA QUE EXISTE NO MUNDO Existe uma lacuna entre a teoria e a prática
% TODO: Melhorar, jogar pra cá alta rotatividade e distribuição heterogênea de
% experiência de desenvolvedores
Existe, no entanto, uma lacuna entre a teoria e a prática do desenvolvimento de
software \cite{glass1996}. Em particular, a difusão de uma boa prática depende
não apenas dos benefícios preconizados por seus entusiastas, mas também de
questões ligadas ao seu uso efetivo, como o custo de implantação e a escassez de
boas ferramentas de apoio. 

% PROBLEMA QUE EXISTE NO ESTADO DA ARTE: contribui pouco para diminuir o gap
% entre teoria e prática porque contribui pouco para entender esse gap.
Por melhor que seja a argumentação, uma prática só é útil quando é efetivamente
adotada no desenvolvimento de um software. Por isso, a compreensão dos fatores
que influenciam essa adoção é uma parte importante do estudo de uma prática de
desenvolvimento. Infelizmente, a maior parte do conhecimento disponível sobre
tais fatores é proveniente de observações isoladas e estudos de caso
\cite{crowston2005}. Essas observações, embora de grande valia, são
essencialmente tendenciosas. Uma compreensão mais abrangente dos fatores
contribuiria para a proposta de práticas de desenvolvimento mais realistas.
% ou para tomar medidas mais efetivas para que as práticas sejam adotadas

%
% ambiente propício advocate

%Infelizmente 
%Embora os benefícios das mais diversas práticas de desenvolvimento de software
%sejam bastante propagandeados, pouco se sabe sobre os fatores que tipicamente
%favorecem a adoção ou o abandono dessas práticas em um projeto de software. A
%maior parte do conhecimento sobre o assunto é proveniente de experiências
%individuais e estudos de caso []. Essas observações, embora de grande valia, são
%essencialmente tendenciosas. 

A escolha feita nesta pesquisa de estudar projetos de software livre certamente
limita a generalidade dos resultados, uma vez que a organização desses projetos
em geral é diferente da organização de projetos comerciais \cite{raymond2001}.
Essa opção se justifica, no entanto, pela importância que tais softwares
adquiriram em diversos segmentos.  Um estudo mostrou que, em 2004, mais de 1
milhão de desenvolvedores no Estados Unidos estavam trabalhando em projetos de
software livre. Além disso, projetos de software livre dominam o mercado mundial
de servidores web, servidores de e-mail e servidores de DNS \cite{wheeler2007}.

%% TODO Ser software livre certamente limita a generalidade dos resultados,
%	% principalmente sabendo que se considera que os processos de
%	% desenvolvimento de software são diferentes daqueles empregados em
%	% softwares comerciais. Nonetheless, acreditamos que o software livre é
%	% tão importante que justifica a importância de um estudo dedicado a ele.
%

%Pesquisa sobre software livre tem se concentrado em estudos de caso
%\cite{mockus2002,capiluppi2007}, com algumas exceções. Crowston e Howison
%\cite{crowston2005} estudaram padrões de comunicação em sistemas de relatório de
%bugs de 120 projetos de software livre e encontraram grande variedade entre os
%projetos. Krishnamurty \cite{krishnamurthy2002} estudou 100 projetos populares e
%percebeu que boa parte só possuía um desenvolvedor.

%Embora pesquisas recentes revelem os processos praticados por alguns projetos
%de software livre em determinado período , não se tem uma descrição do
%cenário mais abrangente, que considera uma amostra significativa de projetos ao
%longo do tempo. 

% TODO: desenvolver a ideia
%Tal descrição permitiria a identificação de dificuldades e limitações na
%aplicação de processos e práticas, contribuindo para a proposta de processos e
%práticas mais realistas.
% Lições práticas a serem estudadas na teoria

%%%%%%%%%%%%%%%%%%%%%%%%%%%%%%%%%%%%%%%%%%%%%%%%%%%%%%%%%%%%%%%%%%%%%%%%%%%%%
%%%%%%%%%%%%%%%%%%%%%%%%%%%%%%%%%%%%%%%%%%%%%%%%%%%%%%%%%%%%%%%%%%%%%%%%%%%%%

\section{Metodologia}

% VIABILIDADE DO ESTUDO
%   Por que software livre? Porque há dados!
%   Restrição a software livre restringe generalidade dos resultados, uma vez que
% em média os processos de sw livre são diferentes de processos de empresas. Mas
% não diminui a importância do estudo, pois sw livre é muito importante!

Muitos projetos de software livre tornam disponíveis publicamente alguns
artefatos produzidos durante o processo de desenvolvimento de software. Esses
artefatos incluem código-fonte e documentação mantidos em sistemas de controle
de versão, bem como relatórios de \emph{bugs} e mensagens de e-mail trocadas
entre desenvolvedores. 

A grande disponibilidade de dados relacionados ao desenvolvimento favorece a
realização de estudos em larga escala. Nesta pesquisa serão utilizados
predominantemente métodos quantitativos de análise de dados para viabilizar o
estudo de uma amostra significativa de projetos de software livre.

A seleção da amostra de projetos será feita com base em informações como número
de desenvolvedores e tamanho do software. Serão extraídos dados de repositórios
de controle de versão, sistemas de acompanhamento de \emph{bugs} e listas de
discussão.  Ferramentas de análise estática de código serão usadas para extrair
relações entre os diversos componentes de um sistema de software.

%Os projetos de software e seus desenvolvedores serão caracterizados com base em
%métricas de redes sociais, como centralidade \cite{crowston2005}, e de outras
%métricas aplicadas em estudos sobre software livre \cite{

A identificação das práticas adotadas por um projeto será feita a partir do uso
de técnicas de mineração de processos \cite{rubin2007} sobre o histórico dos
dados extraídos do projeto. Tais técnicas têm como finalidade extrair modelos de
processos a partir de dados temporais.

Para responder às questões de pesquisa, serão usados métodos de análise
exploratória de dados e inferência estatística, relacionando características de
projetos e de desenvolvedores, o histórico de desenvolvimento e a aplicação de
boas práticas. A partir dos resultados da análise poderão ser realizadas
investigações mais aprofundadas sobre projetos selecionados e comparações com
resultados obtidos em estudos de caso.

%Técnicas de mineração de processos e estatística descritiva
%viabilizam a realização desta pesquisa com grandes amostras de projetos de
%software.
%% TALVEZ SEJA BOM FICAR SÓ COM O ESTUDO QUANTITATIVO
%% O ESTUDO QUALITATIVO PODE SER A PARTE EXPLORATÓRIA DO ESTUDO QUANTITATIVO
%%Esta pesquisa consistirá de um estudo abrangente, envolvendo um grande número
%%de projetos de software livre, e de um estudo mais aprofundado sobre um número
%%reduzido de projetos. Os dois estudos cumprem papéis complementares no
%%entendimento da aplicação de processos em projetos reais.
%
%Para o estudo abrangente será selecionada uma amostra significativa de projetos
%de software livre hospedados em um repositório de código-fonte a ser escolhido,
%como o SourceForge. A amostra será estratificada de acordo com aspectos como
%idade do projeto, número de desenvolvedores e número de linhas de código-fonte.
%
%Serão analisados relatórios de bugs registrados em um sistema de gerenciamento
%de bugs, mensagens de e-mail arquivadas em listas de discussão e arquivos
%mantidos sob controle de versão, como documentação e código-fonte. Ferramentas
%para a extração desses dados já foram usadas em pesquisas anteriores [], com
%bons resultados.
%
%A análise dos dados no estudo mais abrangente será feita de forma automática
%através de técnicas de mineração de processos. Essa abordagem foi escolhida por
%tornar viável a análise de grandes volumes de dados. A mineração de processos
%tem como finalidade de extrair modelos de processos a partir de registros de
%eventos. Eventos que poderão ser extraídos dos dados coletados incluem a
%modificação de um arquivo fonte, o registro da resolução de um bug e o envio de
%uma mensagem à lista de discussão do projeto.
%
%%O algoritmo alfa da mineração de processos é capaz de extrair de uma lista de
%%eventos um modelo de processo representado como uma rede de Petri. Perguntas
%%sobre o modelo extraído podem ser respondidas com o auxílio de checadores de
%%fórmulas da lógica temporal linear (LTL).
%
%Inicialmente será feita uma análise exploratória sobre uma amostra reduzida de
%projetos de software com os métodos e ferramentas escolhidos. Essa análise terá
%por objetivo investigar como os dados devem ser pré-processados e identificar
%limitações da abordagem.
%
%A amostra reduzida de projetos de software também será alvo de um estudo mais
%detalhado. Informações adicionais sobre o processo desses projetos serão
%coletadas através de questionários e entrevistas com desenvolvedores. Quando
%for possível, as informações fornecidas pelos desenvolvedores serão
%confrontadas com as informações extraídas automaticamente. 


%%%%%%%%%%%%%%%%%%%%%%%%%%%%%%%%%%%%%%%%%%%%%%%%%%%%%%%%%%%%%%%%%%%%%%%%%%%%%
%%%%%%%%%%%%%%%%%%%%%%%%%%%%%%%%%%%%%%%%%%%%%%%%%%%%%%%%%%%%%%%%%%%%%%%%%%%%%

\section{Proposta de Cronograma}

As seguintes atividades serão realizados durante este projeto, de acordo com os
prazos indicados na Tabela \ref{tab:cronograma}:

\begin{enumerate}
  \item \label{prevista:estudos}
    \textbf{Revisão bibliográfica}. Serão lidos textos sobre engenharia de software
    experimental, processos e práticas de desenvolvimento de software,
    desenvolvimento de software livre, mineração de processos e estatística,
    entre outros. Ao final será produzido um resumo dos tópicos mais relevantes.
%  \item \label{prevista:artigos}
%    Produção de artigos. Os artigos serão submetidos para veículos de publicação
%    a serem definidos.
  \item \label{prevista:qualificacao}
    \textbf{Realização do exame de qualificação}.
  \item \label{prevista:proposta}
    \textbf{Defesa da proposta de tese}.
  \item \label{prevista:experimentos}
    \textbf{Planejamento e execução de experimentos}. Espera-se produzir pelo menos dois
    artigos: um com resultados preliminares e outro com os resultados finais dos
    experimentos.
  \item \label{prevista:redacao}
    \textbf{Redação da tese}.
  \item \label{prevista:defesa}
    \textbf{Defesa da tese}.
\end{enumerate}

\begin{table}[h]
  \centering
  \begin{tabular}{|c|c|c|c|c|c|c|c|c|} \hline
    Atividade                   & 2010.1 & 2010.2 & 2011.1 & 2011.2  & 2012.1 & 2012.2  & 2013.1 & 2013.2 \newrow
    \ref{prevista:estudos}      & \x     & \x     & \x     &         &        &         &        &        \newrow
%    \ref{prevista:projeto}      &        &        &        &         &        &         &        &        \newrow
%    \ref{prevista:artigos}      &        &        &        & \x      &        & \x      &        &        \newrow
    \ref{prevista:qualificacao} &        &        & \x     &         &        &         &        &        \newrow
    \ref{prevista:proposta}     &        &        &        & \x      &        &         &        &        \newrow
    \ref{prevista:experimentos} &        &        &        & \x      & \x     & \x      &        &        \newrow
    \ref{prevista:redacao}      &        &        &        &         &        & \x      & \x     & \x     \newrow
    \ref{prevista:defesa}       &        &        &        &         &        &         &        & \x     \newrow
  \end{tabular}
 \caption{Proposta de Cronograma de Atividades}
 \label{tab:cronograma}
\end{table}

%%%%%%%%%%%%%%%%%%%%%%%%%%%%%%%%%%%%%%%%%%%%%%%%%%%%%%%%%%%%%%%%%%%%%%%%%%%%%
%%%%%%%%%%%%%%%%%%%%%%%%%%%%%%%%%%%%%%%%%%%%%%%%%%%%%%%%%%%%%%%%%%%%%%%%%%%%%

\bibliographystyle{plain}
\bibliography{doutorado}

\end{document}
